\documentclass[11pt]{article}

% \usepackage{natbib}

\usepackage{titlesec}
\setcounter{secnumdepth}{4}
\titleformat{\paragraph}
{\normalfont\normalsize\bfseries}{\theparagraph}{1em}{}
\titlespacing*{\paragraph}
{0pt}{3.25ex plus 1ex minus .2ex}{1.5ex plus .2ex}


\usepackage[square,numbers]{natbib}
\bibliographystyle{abbrvnat}


\usepackage[breakable]{tcolorbox}
\usepackage{parskip} % Stop auto-indenting (to mimic markdown behaviour)

\usepackage{iftex}
\ifPDFTeX
	\usepackage[T1]{fontenc}
	\usepackage{mathpazo}
\else
	\usepackage{fontspec}
\fi

% Basic figure setup, for now with no caption control since it's done
% automatically by Pandoc (which extracts ![](path) syntax from Markdown).
\usepackage{graphicx}
% Maintain compatibility with old templates. Remove in nbconvert 6.0
\let\Oldincludegraphics\includegraphics
% Ensure that by default, figures have no caption (until we provide a
% proper Figure object with a Caption API and a way to capture that
% in the conversion process - todo).
\usepackage{caption}
\DeclareCaptionFormat{nocaption}{}
\captionsetup{format=nocaption,aboveskip=0pt,belowskip=0pt}

\usepackage[Export]{adjustbox} % Used to constrain images to a maximum size
\adjustboxset{max size={0.9\linewidth}{0.9\paperheight}}
\usepackage{float}
\floatplacement{figure}{H} % forces figures to be placed at the correct location
\usepackage{xcolor} % Allow colors to be defined
\usepackage{enumerate} % Needed for markdown enumerations to work
\usepackage{geometry} % Used to adjust the document margins
\usepackage{amsmath} % Equations
\usepackage{amssymb} % Equations
\usepackage{textcomp} % defines textquotesingle
% Hack from http://tex.stackexchange.com/a/47451/13684:
\AtBeginDocument{%
    \def\PYZsq{\textquotesingle}% Upright quotes in Pygmentized code
}
\usepackage{upquote} % Upright quotes for verbatim code
\usepackage{eurosym} % defines \euro
% \usepackage[mathletters]{ucs} % Extended unicode (utf-8) support
\usepackage{fancyvrb} % verbatim replacement that allows latex
\usepackage{grffile} % extends the file name processing of package graphics 
                     % to support a larger range
\makeatletter % fix for grffile with XeLaTeX
\def\Gread@@xetex#1{%
  \IfFileExists{"\Gin@base".bb}%
  {\Gread@eps{\Gin@base.bb}}%
  {\Gread@@xetex@aux#1}%
}
\makeatother

% The hyperref package gives us a pdf with properly built
% internal navigation ('pdf bookmarks' for the table of contents,
% internal cross-reference links, web links for URLs, etc.)
\usepackage{hyperref}
% The default LaTeX title has an obnoxious amount of whitespace. By default,
% titling removes some of it. It also provides customization options.
\usepackage{titling}
\usepackage{longtable} % longtable support required by pandoc >1.10
\usepackage{booktabs}  % table support for pandoc > 1.12.2
\usepackage[inline]{enumitem} % IRkernel/repr support (it uses the enumerate* environment)
\usepackage[normalem]{ulem} % ulem is needed to support strikethroughs (\sout)
                            % normalem makes italics be italics, not underlines
\usepackage{mathrsfs}



% Colors for the hyperref package
\definecolor{urlcolor}{rgb}{0,.145,.698}
\definecolor{linkcolor}{rgb}{.71,0.21,0.01}
\definecolor{citecolor}{rgb}{.12,.54,.11}

% ANSI colors
\definecolor{ansi-black}{HTML}{3E424D}
\definecolor{ansi-black-intense}{HTML}{282C36}
\definecolor{ansi-red}{HTML}{E75C58}
\definecolor{ansi-red-intense}{HTML}{B22B31}
\definecolor{ansi-green}{HTML}{00A250}
\definecolor{ansi-green-intense}{HTML}{007427}
\definecolor{ansi-yellow}{HTML}{DDB62B}
\definecolor{ansi-yellow-intense}{HTML}{B27D12}
\definecolor{ansi-blue}{HTML}{208FFB}
\definecolor{ansi-blue-intense}{HTML}{0065CA}
\definecolor{ansi-magenta}{HTML}{D160C4}
\definecolor{ansi-magenta-intense}{HTML}{A03196}
\definecolor{ansi-cyan}{HTML}{60C6C8}
\definecolor{ansi-cyan-intense}{HTML}{258F8F}
\definecolor{ansi-white}{HTML}{C5C1B4}
\definecolor{ansi-white-intense}{HTML}{A1A6B2}
\definecolor{ansi-default-inverse-fg}{HTML}{FFFFFF}
\definecolor{ansi-default-inverse-bg}{HTML}{000000}

% commands and environments needed by pandoc snippets
% extracted from the output of `pandoc -s`
\providecommand{\tightlist}{%
  \setlength{\itemsep}{0pt}\setlength{\parskip}{0pt}}
\DefineVerbatimEnvironment{Highlighting}{Verbatim}{commandchars=\\\{\}}
% Add ',fontsize=\small' for more characters per line
\newenvironment{Shaded}{}{}
\newcommand{\KeywordTok}[1]{\textcolor[rgb]{0.00,0.44,0.13}{\textbf{{#1}}}}
\newcommand{\DataTypeTok}[1]{\textcolor[rgb]{0.56,0.13,0.00}{{#1}}}
\newcommand{\DecValTok}[1]{\textcolor[rgb]{0.25,0.63,0.44}{{#1}}}
\newcommand{\BaseNTok}[1]{\textcolor[rgb]{0.25,0.63,0.44}{{#1}}}
\newcommand{\FloatTok}[1]{\textcolor[rgb]{0.25,0.63,0.44}{{#1}}}
\newcommand{\CharTok}[1]{\textcolor[rgb]{0.25,0.44,0.63}{{#1}}}
\newcommand{\StringTok}[1]{\textcolor[rgb]{0.25,0.44,0.63}{{#1}}}
\newcommand{\CommentTok}[1]{\textcolor[rgb]{0.38,0.63,0.69}{\textit{{#1}}}}
\newcommand{\OtherTok}[1]{\textcolor[rgb]{0.00,0.44,0.13}{{#1}}}
\newcommand{\AlertTok}[1]{\textcolor[rgb]{1.00,0.00,0.00}{\textbf{{#1}}}}
\newcommand{\FunctionTok}[1]{\textcolor[rgb]{0.02,0.16,0.49}{{#1}}}
\newcommand{\RegionMarkerTok}[1]{{#1}}
\newcommand{\ErrorTok}[1]{\textcolor[rgb]{1.00,0.00,0.00}{\textbf{{#1}}}}
\newcommand{\NormalTok}[1]{{#1}}

% Additional commands for more recent versions of Pandoc
\newcommand{\ConstantTok}[1]{\textcolor[rgb]{0.53,0.00,0.00}{{#1}}}
\newcommand{\SpecialCharTok}[1]{\textcolor[rgb]{0.25,0.44,0.63}{{#1}}}
\newcommand{\VerbatimStringTok}[1]{\textcolor[rgb]{0.25,0.44,0.63}{{#1}}}
\newcommand{\SpecialStringTok}[1]{\textcolor[rgb]{0.73,0.40,0.53}{{#1}}}
\newcommand{\ImportTok}[1]{{#1}}
\newcommand{\DocumentationTok}[1]{\textcolor[rgb]{0.73,0.13,0.13}{\textit{{#1}}}}
\newcommand{\AnnotationTok}[1]{\textcolor[rgb]{0.38,0.63,0.69}{\textbf{\textit{{#1}}}}}
\newcommand{\CommentVarTok}[1]{\textcolor[rgb]{0.38,0.63,0.69}{\textbf{\textit{{#1}}}}}
\newcommand{\VariableTok}[1]{\textcolor[rgb]{0.10,0.09,0.49}{{#1}}}
\newcommand{\ControlFlowTok}[1]{\textcolor[rgb]{0.00,0.44,0.13}{\textbf{{#1}}}}
\newcommand{\OperatorTok}[1]{\textcolor[rgb]{0.40,0.40,0.40}{{#1}}}
\newcommand{\BuiltInTok}[1]{{#1}}
\newcommand{\ExtensionTok}[1]{{#1}}
\newcommand{\PreprocessorTok}[1]{\textcolor[rgb]{0.74,0.48,0.00}{{#1}}}
\newcommand{\AttributeTok}[1]{\textcolor[rgb]{0.49,0.56,0.16}{{#1}}}
\newcommand{\InformationTok}[1]{\textcolor[rgb]{0.38,0.63,0.69}{\textbf{\textit{{#1}}}}}
\newcommand{\WarningTok}[1]{\textcolor[rgb]{0.38,0.63,0.69}{\textbf{\textit{{#1}}}}}


% Define a nice break command that doesn't care if a line doesn't already
% exist.
\def\br{\hspace*{\fill} \\* }
% Math Jax compatibility definitions
\def\gt{>}
\def\lt{<}
\let\Oldtex\TeX
\let\Oldlatex\LaTeX
\renewcommand{\TeX}{\textrm{\Oldtex}}
\renewcommand{\LaTeX}{\textrm{\Oldlatex}}
% Document parameters
% Document title
% \title{Final Project}
    
    
    
    
    
% Pygments definitions
\makeatletter
\def\PY@reset{\let\PY@it=\relax \let\PY@bf=\relax%
    \let\PY@ul=\relax \let\PY@tc=\relax%
    \let\PY@bc=\relax \let\PY@ff=\relax}
\def\PY@tok#1{\csname PY@tok@#1\endcsname}
\def\PY@toks#1+{\ifx\relax#1\empty\else%
    \PY@tok{#1}\expandafter\PY@toks\fi}
\def\PY@do#1{\PY@bc{\PY@tc{\PY@ul{%
    \PY@it{\PY@bf{\PY@ff{#1}}}}}}}
\def\PY#1#2{\PY@reset\PY@toks#1+\relax+\PY@do{#2}}

\@namedef{PY@tok@w}{\def\PY@tc##1{\textcolor[rgb]{0.73,0.73,0.73}{##1}}}
\@namedef{PY@tok@c}{\let\PY@it=\textit\def\PY@tc##1{\textcolor[rgb]{0.25,0.50,0.50}{##1}}}
\@namedef{PY@tok@cp}{\def\PY@tc##1{\textcolor[rgb]{0.74,0.48,0.00}{##1}}}
\@namedef{PY@tok@k}{\let\PY@bf=\textbf\def\PY@tc##1{\textcolor[rgb]{0.00,0.50,0.00}{##1}}}
\@namedef{PY@tok@kp}{\def\PY@tc##1{\textcolor[rgb]{0.00,0.50,0.00}{##1}}}
\@namedef{PY@tok@kt}{\def\PY@tc##1{\textcolor[rgb]{0.69,0.00,0.25}{##1}}}
\@namedef{PY@tok@o}{\def\PY@tc##1{\textcolor[rgb]{0.40,0.40,0.40}{##1}}}
\@namedef{PY@tok@ow}{\let\PY@bf=\textbf\def\PY@tc##1{\textcolor[rgb]{0.67,0.13,1.00}{##1}}}
\@namedef{PY@tok@nb}{\def\PY@tc##1{\textcolor[rgb]{0.00,0.50,0.00}{##1}}}
\@namedef{PY@tok@nf}{\def\PY@tc##1{\textcolor[rgb]{0.00,0.00,1.00}{##1}}}
\@namedef{PY@tok@nc}{\let\PY@bf=\textbf\def\PY@tc##1{\textcolor[rgb]{0.00,0.00,1.00}{##1}}}
\@namedef{PY@tok@nn}{\let\PY@bf=\textbf\def\PY@tc##1{\textcolor[rgb]{0.00,0.00,1.00}{##1}}}
\@namedef{PY@tok@ne}{\let\PY@bf=\textbf\def\PY@tc##1{\textcolor[rgb]{0.82,0.25,0.23}{##1}}}
\@namedef{PY@tok@nv}{\def\PY@tc##1{\textcolor[rgb]{0.10,0.09,0.49}{##1}}}
\@namedef{PY@tok@no}{\def\PY@tc##1{\textcolor[rgb]{0.53,0.00,0.00}{##1}}}
\@namedef{PY@tok@nl}{\def\PY@tc##1{\textcolor[rgb]{0.63,0.63,0.00}{##1}}}
\@namedef{PY@tok@ni}{\let\PY@bf=\textbf\def\PY@tc##1{\textcolor[rgb]{0.60,0.60,0.60}{##1}}}
\@namedef{PY@tok@na}{\def\PY@tc##1{\textcolor[rgb]{0.49,0.56,0.16}{##1}}}
\@namedef{PY@tok@nt}{\let\PY@bf=\textbf\def\PY@tc##1{\textcolor[rgb]{0.00,0.50,0.00}{##1}}}
\@namedef{PY@tok@nd}{\def\PY@tc##1{\textcolor[rgb]{0.67,0.13,1.00}{##1}}}
\@namedef{PY@tok@s}{\def\PY@tc##1{\textcolor[rgb]{0.73,0.13,0.13}{##1}}}
\@namedef{PY@tok@sd}{\let\PY@it=\textit\def\PY@tc##1{\textcolor[rgb]{0.73,0.13,0.13}{##1}}}
\@namedef{PY@tok@si}{\let\PY@bf=\textbf\def\PY@tc##1{\textcolor[rgb]{0.73,0.40,0.53}{##1}}}
\@namedef{PY@tok@se}{\let\PY@bf=\textbf\def\PY@tc##1{\textcolor[rgb]{0.73,0.40,0.13}{##1}}}
\@namedef{PY@tok@sr}{\def\PY@tc##1{\textcolor[rgb]{0.73,0.40,0.53}{##1}}}
\@namedef{PY@tok@ss}{\def\PY@tc##1{\textcolor[rgb]{0.10,0.09,0.49}{##1}}}
\@namedef{PY@tok@sx}{\def\PY@tc##1{\textcolor[rgb]{0.00,0.50,0.00}{##1}}}
\@namedef{PY@tok@m}{\def\PY@tc##1{\textcolor[rgb]{0.40,0.40,0.40}{##1}}}
\@namedef{PY@tok@gh}{\let\PY@bf=\textbf\def\PY@tc##1{\textcolor[rgb]{0.00,0.00,0.50}{##1}}}
\@namedef{PY@tok@gu}{\let\PY@bf=\textbf\def\PY@tc##1{\textcolor[rgb]{0.50,0.00,0.50}{##1}}}
\@namedef{PY@tok@gd}{\def\PY@tc##1{\textcolor[rgb]{0.63,0.00,0.00}{##1}}}
\@namedef{PY@tok@gi}{\def\PY@tc##1{\textcolor[rgb]{0.00,0.63,0.00}{##1}}}
\@namedef{PY@tok@gr}{\def\PY@tc##1{\textcolor[rgb]{1.00,0.00,0.00}{##1}}}
\@namedef{PY@tok@ge}{\let\PY@it=\textit}
\@namedef{PY@tok@gs}{\let\PY@bf=\textbf}
\@namedef{PY@tok@gp}{\let\PY@bf=\textbf\def\PY@tc##1{\textcolor[rgb]{0.00,0.00,0.50}{##1}}}
\@namedef{PY@tok@go}{\def\PY@tc##1{\textcolor[rgb]{0.53,0.53,0.53}{##1}}}
\@namedef{PY@tok@gt}{\def\PY@tc##1{\textcolor[rgb]{0.00,0.27,0.87}{##1}}}
\@namedef{PY@tok@err}{\def\PY@bc##1{{\setlength{\fboxsep}{\string -\fboxrule}\fcolorbox[rgb]{1.00,0.00,0.00}{1,1,1}{\strut ##1}}}}
\@namedef{PY@tok@kc}{\let\PY@bf=\textbf\def\PY@tc##1{\textcolor[rgb]{0.00,0.50,0.00}{##1}}}
\@namedef{PY@tok@kd}{\let\PY@bf=\textbf\def\PY@tc##1{\textcolor[rgb]{0.00,0.50,0.00}{##1}}}
\@namedef{PY@tok@kn}{\let\PY@bf=\textbf\def\PY@tc##1{\textcolor[rgb]{0.00,0.50,0.00}{##1}}}
\@namedef{PY@tok@kr}{\let\PY@bf=\textbf\def\PY@tc##1{\textcolor[rgb]{0.00,0.50,0.00}{##1}}}
\@namedef{PY@tok@bp}{\def\PY@tc##1{\textcolor[rgb]{0.00,0.50,0.00}{##1}}}
\@namedef{PY@tok@fm}{\def\PY@tc##1{\textcolor[rgb]{0.00,0.00,1.00}{##1}}}
\@namedef{PY@tok@vc}{\def\PY@tc##1{\textcolor[rgb]{0.10,0.09,0.49}{##1}}}
\@namedef{PY@tok@vg}{\def\PY@tc##1{\textcolor[rgb]{0.10,0.09,0.49}{##1}}}
\@namedef{PY@tok@vi}{\def\PY@tc##1{\textcolor[rgb]{0.10,0.09,0.49}{##1}}}
\@namedef{PY@tok@vm}{\def\PY@tc##1{\textcolor[rgb]{0.10,0.09,0.49}{##1}}}
\@namedef{PY@tok@sa}{\def\PY@tc##1{\textcolor[rgb]{0.73,0.13,0.13}{##1}}}
\@namedef{PY@tok@sb}{\def\PY@tc##1{\textcolor[rgb]{0.73,0.13,0.13}{##1}}}
\@namedef{PY@tok@sc}{\def\PY@tc##1{\textcolor[rgb]{0.73,0.13,0.13}{##1}}}
\@namedef{PY@tok@dl}{\def\PY@tc##1{\textcolor[rgb]{0.73,0.13,0.13}{##1}}}
\@namedef{PY@tok@s2}{\def\PY@tc##1{\textcolor[rgb]{0.73,0.13,0.13}{##1}}}
\@namedef{PY@tok@sh}{\def\PY@tc##1{\textcolor[rgb]{0.73,0.13,0.13}{##1}}}
\@namedef{PY@tok@s1}{\def\PY@tc##1{\textcolor[rgb]{0.73,0.13,0.13}{##1}}}
\@namedef{PY@tok@mb}{\def\PY@tc##1{\textcolor[rgb]{0.40,0.40,0.40}{##1}}}
\@namedef{PY@tok@mf}{\def\PY@tc##1{\textcolor[rgb]{0.40,0.40,0.40}{##1}}}
\@namedef{PY@tok@mh}{\def\PY@tc##1{\textcolor[rgb]{0.40,0.40,0.40}{##1}}}
\@namedef{PY@tok@mi}{\def\PY@tc##1{\textcolor[rgb]{0.40,0.40,0.40}{##1}}}
\@namedef{PY@tok@il}{\def\PY@tc##1{\textcolor[rgb]{0.40,0.40,0.40}{##1}}}
\@namedef{PY@tok@mo}{\def\PY@tc##1{\textcolor[rgb]{0.40,0.40,0.40}{##1}}}
\@namedef{PY@tok@ch}{\let\PY@it=\textit\def\PY@tc##1{\textcolor[rgb]{0.25,0.50,0.50}{##1}}}
\@namedef{PY@tok@cm}{\let\PY@it=\textit\def\PY@tc##1{\textcolor[rgb]{0.25,0.50,0.50}{##1}}}
\@namedef{PY@tok@cpf}{\let\PY@it=\textit\def\PY@tc##1{\textcolor[rgb]{0.25,0.50,0.50}{##1}}}
\@namedef{PY@tok@c1}{\let\PY@it=\textit\def\PY@tc##1{\textcolor[rgb]{0.25,0.50,0.50}{##1}}}
\@namedef{PY@tok@cs}{\let\PY@it=\textit\def\PY@tc##1{\textcolor[rgb]{0.25,0.50,0.50}{##1}}}

\def\PYZbs{\char`\\}
\def\PYZus{\char`\_}
\def\PYZob{\char`\{}
\def\PYZcb{\char`\}}
\def\PYZca{\char`\^}
\def\PYZam{\char`\&}
\def\PYZlt{\char`\<}
\def\PYZgt{\char`\>}
\def\PYZsh{\char`\#}
\def\PYZpc{\char`\%}
\def\PYZdl{\char`\$}
\def\PYZhy{\char`\-}
\def\PYZsq{\char`\'}
\def\PYZdq{\char`\"}
\def\PYZti{\char`\~}
% for compatibility with earlier versions
\def\PYZat{@}
\def\PYZlb{[}
\def\PYZrb{]}
\makeatother


    % For linebreaks inside Verbatim environment from package fancyvrb. 
    \makeatletter
        \newbox\Wrappedcontinuationbox 
        \newbox\Wrappedvisiblespacebox 
        \newcommand*\Wrappedvisiblespace {\textcolor{red}{\textvisiblespace}} 
        \newcommand*\Wrappedcontinuationsymbol {\textcolor{red}{\llap{\tiny$\m@th\hookrightarrow$}}} 
        \newcommand*\Wrappedcontinuationindent {3ex } 
        \newcommand*\Wrappedafterbreak {\kern\Wrappedcontinuationindent\copy\Wrappedcontinuationbox} 
        % Take advantage of the already applied Pygments mark-up to insert 
        % potential linebreaks for TeX processing. 
        %        {, <, #, %, $, ' and ": go to next line. 
        %        _, }, ^, &, >, - and ~: stay at end of broken line. 
        % Use of \textquotesingle for straight quote. 
        \newcommand*\Wrappedbreaksatspecials {% 
            \def\PYGZus{\discretionary{\char`\_}{\Wrappedafterbreak}{\char`\_}}% 
            \def\PYGZob{\discretionary{}{\Wrappedafterbreak\char`\{}{\char`\{}}% 
            \def\PYGZcb{\discretionary{\char`\}}{\Wrappedafterbreak}{\char`\}}}% 
            \def\PYGZca{\discretionary{\char`\^}{\Wrappedafterbreak}{\char`\^}}% 
            \def\PYGZam{\discretionary{\char`\&}{\Wrappedafterbreak}{\char`\&}}% 
            \def\PYGZlt{\discretionary{}{\Wrappedafterbreak\char`\<}{\char`\<}}% 
            \def\PYGZgt{\discretionary{\char`\>}{\Wrappedafterbreak}{\char`\>}}% 
            \def\PYGZsh{\discretionary{}{\Wrappedafterbreak\char`\#}{\char`\#}}% 
            \def\PYGZpc{\discretionary{}{\Wrappedafterbreak\char`\%}{\char`\%}}% 
            \def\PYGZdl{\discretionary{}{\Wrappedafterbreak\char`\$}{\char`\$}}% 
            \def\PYGZhy{\discretionary{\char`\-}{\Wrappedafterbreak}{\char`\-}}% 
            \def\PYGZsq{\discretionary{}{\Wrappedafterbreak\textquotesingle}{\textquotesingle}}% 
            \def\PYGZdq{\discretionary{}{\Wrappedafterbreak\char`\"}{\char`\"}}% 
            \def\PYGZti{\discretionary{\char`\~}{\Wrappedafterbreak}{\char`\~}}% 
        } 
        % Some characters . , ; ? ! / are not pygmentized. 
        % This macro makes them "active" and they will insert potential linebreaks 
        \newcommand*\Wrappedbreaksatpunct {% 
            \lccode`\~`\.\lowercase{\def~}{\discretionary{\hbox{\char`\.}}{\Wrappedafterbreak}{\hbox{\char`\.}}}% 
            \lccode`\~`\,\lowercase{\def~}{\discretionary{\hbox{\char`\,}}{\Wrappedafterbreak}{\hbox{\char`\,}}}% 
            \lccode`\~`\;\lowercase{\def~}{\discretionary{\hbox{\char`\;}}{\Wrappedafterbreak}{\hbox{\char`\;}}}% 
            \lccode`\~`\:\lowercase{\def~}{\discretionary{\hbox{\char`\:}}{\Wrappedafterbreak}{\hbox{\char`\:}}}% 
            \lccode`\~`\?\lowercase{\def~}{\discretionary{\hbox{\char`\?}}{\Wrappedafterbreak}{\hbox{\char`\?}}}% 
            \lccode`\~`\!\lowercase{\def~}{\discretionary{\hbox{\char`\!}}{\Wrappedafterbreak}{\hbox{\char`\!}}}% 
            \lccode`\~`\/\lowercase{\def~}{\discretionary{\hbox{\char`\/}}{\Wrappedafterbreak}{\hbox{\char`\/}}}% 
            \catcode`\.\active
            \catcode`\,\active 
            \catcode`\;\active
            \catcode`\:\active
            \catcode`\?\active
            \catcode`\!\active
            \catcode`\/\active 
            \lccode`\~`\~ 	
        }
    \makeatother

    \let\OriginalVerbatim=\Verbatim
    \makeatletter
    \renewcommand{\Verbatim}[1][1]{%
        %\parskip\z@skip
        \sbox\Wrappedcontinuationbox {\Wrappedcontinuationsymbol}%
        \sbox\Wrappedvisiblespacebox {\FV@SetupFont\Wrappedvisiblespace}%
        \def\FancyVerbFormatLine ##1{\hsize\linewidth
            \vtop{\raggedright\hyphenpenalty\z@\exhyphenpenalty\z@
                \doublehyphendemerits\z@\finalhyphendemerits\z@
                \strut ##1\strut}%
        }%
        % If the linebreak is at a space, the latter will be displayed as visible
        % space at end of first line, and a continuation symbol starts next line.
        % Stretch/shrink are however usually zero for typewriter font.
        \def\FV@Space {%
            \nobreak\hskip\z@ plus\fontdimen3\font minus\fontdimen4\font
            \discretionary{\copy\Wrappedvisiblespacebox}{\Wrappedafterbreak}
            {\kern\fontdimen2\font}%
        }%
        
        % Allow breaks at special characters using \PYG... macros.
        \Wrappedbreaksatspecials
        % Breaks at punctuation characters . , ; ? ! and / need catcode=\active 	
        \OriginalVerbatim[#1,codes*=\Wrappedbreaksatpunct]%
    }
    \makeatother

    % Exact colors from NB
    \definecolor{incolor}{HTML}{303F9F}
    \definecolor{outcolor}{HTML}{D84315}
    \definecolor{cellborder}{HTML}{CFCFCF}
    \definecolor{cellbackground}{HTML}{F7F7F7}
    
    % prompt
    \makeatletter
    \newcommand{\boxspacing}{\kern\kvtcb@left@rule\kern\kvtcb@boxsep}
    \makeatother
    \newcommand{\prompt}[4]{
        \ttfamily\llap{{\color{#2}[#3]:\hspace{3pt}#4}}\vspace{-\baselineskip}
    }
    

    
    % Prevent overflowing lines due to hard-to-break entities
    \sloppy 
    % Setup hyperref package
    \hypersetup{
      breaklinks=true,  % so long urls are correctly broken across lines
      colorlinks=true,
      urlcolor=urlcolor,
      linkcolor=linkcolor,
      citecolor=citecolor,
      }
    % Slightly bigger margins than the latex defaults
    
    \geometry{verbose,tmargin=1in,bmargin=1in,lmargin=1in,rmargin=1in}

\setlength\parindent{0pt}
\setlength\parskip{\medskipamount}

\title{Feasibility and Impact of Converting UA's Surface Parking to
Solar
Power}

\begin{document}
    
\begin{titlepage}
    \begin{center}
        \vspace*{1cm}
            
        \Huge
        \textbf{Feasibility and Impact of Converting UA's Surface Parking to Solar Power}
            
        \vspace{0.5cm}
        \LARGE
        Final Project for ME 591 - Sustainable Energy
            
        \vspace{1.5cm}
            
        \textbf{Max Schrader}
            
        \vfill
            
        \vspace{0.8cm}
            
        \includegraphics[width=0.4\textwidth]{university}
            
        \Large
        Mechanical Engineering\\
        University of Alabama\\
        USA\\
        \today
            
    \end{center}
\end{titlepage}

\hypertarget{introduction}{%
\section{Introduction}\label{introduction}}

To satisfy the ME-591 Sustainable Energy Project, students were asked to
analyze a sustainable energy project. As it is an extremely open-ended
topic, the professor gave a project centering on the University of
Alabama (UA) as an example, and recommended to other students a project on powering UA buses with hydrogen gas.

Using UA as my centering object, my proposed project is powering UA's
main campus with solar power. The land used for the solar panels will be
the non-parking garage parking on campus. The parking lots were chosen for several reasons, but one was so that the scale of the solar project could be compared to land plots in our frame of reference. In reality, if UA were to use solar, they would likely buy a cheap plot of land outside of city limits.

The potential negatives of replacing parking with solar are
initially glaring, i.e.~where will people park? As such, I think it is
important to address the parking fears before moving onto the energy
analysis and economics. Replacing all of UA's \emph{commuter} surface
parking would reduce the amount of campus parking by \textbf{6,136
spots} (per email correspondence with the Transportation Services
Department), which is a substantial amount. 

In the time that I have been at UA (since 2015), multiple new parking lots have been constructed, showing that UA prioritizes on-campus parking. I think that this is a
critical mistake. There is no data published by UA on the distance that
the average student who parks on-campus commutes, but I would contend
from personal experience that it is not far. While it is not in the
scope of this project, I think interesting research could be done on the
environmental benefit of replacing commuter trips with bus routes -
buses that are used because an alternative is not available. In addition
to the environmental benefits, built places centered on public transit and
foot-traffic are generally more enjoyable. Anyone who travels to Europe
notices that the best places also seem to be the hardest to get to via
car. I would contend that there is correlation, and UA should experiment
with designing a mostly car-free environment. It would be extremely
hard to get faculty and staff on board with a transition to a
"car-free" campus if their parking was eliminated, so parking garages in my plan
would remain open, serving both staff as well as ADA parking and UA
students who commutes a over a specific distance.

With the reduction in parking addressed (whether it was a sufficient
argument or not being beyond the scope of this project), the remaining
pages of this report cover the economics and environmental benefit of
converting UA's surface parking to solar panel arrays. The first
iteration of this report was created with the naive assumption that
solar could power the entire campus, and thus calculations were also
done for a hydro battery. After receiving UA's energy consumption data
for an entire year, it was clear the converting the parking lots to
solar arrays could not supply all of UA's power, instead it could only
ever make up about 13\% of the power requirements. This means that
energy storage was not necessary, and therefore the calculations are not
included in this iteration of the report.

A table summarizing the findings of this report can be seen below, where most of the values are given on a yearly basis. The "Solar" row highlights the metrics if solar were to be implemented, and the "No Solar" row can be taken as the current state of the University of Alabama.

{\footnotesize
\begin{tabular}{llllllll}
% \small
% \addtolength{\tabcolsep}{-1pt}
\toprule
{} &    Cost [\$] & Electric Cost [\$] & CO2 [Mton] & Energy [GWh] & Power [MW] & Payback Per. &    ROI [\%] \\
\midrule
No Solar &           - &               33,034,052 &        39,690,322 &                          244 &                         28 &              - &          - \\
Solar    &  21,064,188 &               25,186,087 &        34,394,988 &                          212 &                         24 &           6.25 &  19.65 \\
\bottomrule
\end{tabular}
}

\emph{It's important to know that I generated below using Jupyter, Python and some Latex. It is likely un-orthodox, but conceivably one could copy my code below and check the results themselves. If you wish to run the code, I can supply the data sources.}

\hypertarget{calculations}{%
\section{Calculations}\label{calculations}}

\hypertarget{python-imports}{%
\subsection{Python Imports}\label{python-imports}}

\begin{tcolorbox}[breakable, size=fbox, boxrule=1pt, pad at break*=1mm,colback=cellbackground, colframe=cellborder]
\prompt{In}{incolor}{1}{\boxspacing}
\begin{Verbatim}[commandchars=\\\{\}]
\PY{k+kn}{import} \PY{n+nn}{pint}
\PY{k+kn}{import} \PY{n+nn}{math}
\PY{k+kn}{import} \PY{n+nn}{numpy} \PY{k}{as} \PY{n+nn}{np}
\PY{k+kn}{import} \PY{n+nn}{pandas} \PY{k}{as} \PY{n+nn}{pd}
\PY{k+kn}{from} \PY{n+nn}{pytz} \PY{k+kn}{import} \PY{n}{timezone}
\PY{k+kn}{import} \PY{n+nn}{pint\PYZus{}pandas}
\PY{k+kn}{import} \PY{n+nn}{atlite}
\PY{k+kn}{import} \PY{n+nn}{logging}
\PY{k+kn}{import} \PY{n+nn}{geopandas} \PY{k}{as} \PY{n+nn}{gpd}
\PY{k+kn}{import} \PY{n+nn}{plotly}\PY{n+nn}{.}\PY{n+nn}{graph\PYZus{}objects} \PY{k}{as} \PY{n+nn}{go}
\PY{k+kn}{from} \PY{n+nn}{plotly}\PY{n+nn}{.}\PY{n+nn}{subplots} \PY{k+kn}{import} \PY{n}{make\PYZus{}subplots}
\PY{k+kn}{import} \PY{n+nn}{plotly}\PY{n+nn}{.}\PY{n+nn}{io} \PY{k}{as} \PY{n+nn}{pio}

\PY{n}{pio}\PY{o}{.}\PY{n}{templates}\PY{o}{.}\PY{n}{default} \PY{o}{=} \PY{l+s+s2}{\PYZdq{}}\PY{l+s+s2}{ggplot2}\PY{l+s+s2}{\PYZdq{}}
\PY{n}{pio}\PY{o}{.}\PY{n}{renderers}\PY{o}{.}\PY{n}{default} \PY{o}{=} \PY{l+s+s2}{\PYZdq{}}\PY{l+s+s2}{notebook+pdf}\PY{l+s+s2}{\PYZdq{}}
\PY{n}{logging}\PY{o}{.}\PY{n}{basicConfig}\PY{p}{(}\PY{n}{level}\PY{o}{=}\PY{n}{logging}\PY{o}{.}\PY{n}{ERROR}\PY{p}{)}
\PY{n}{ureg} \PY{o}{=} \PY{n}{pint}\PY{o}{.}\PY{n}{UnitRegistry}\PY{p}{(}\PY{n}{autoconvert\PYZus{}offset\PYZus{}to\PYZus{}baseunit} \PY{o}{=} \PY{k+kc}{True}\PY{p}{)}
\PY{n}{pint\PYZus{}pandas}\PY{o}{.}\PY{n}{PintType}\PY{o}{.}\PY{n}{ureg} \PY{o}{=} \PY{n}{ureg}
\end{Verbatim}
\end{tcolorbox}


\hypertarget{assessing-uas-current-electricity-demand}{%
\subsection{Assessing UA's Current Electricity
Demand}\label{assessing-uas-current-electricity-demand}}

To compare the solar potential of UA's parking lots to reality, I reached out to Energy Engineering department at UA. They graciously agreed to share UA's electricity consumption data in great detail. The data comes specifically from Will Stephens, an Energy Engineer with UA's
Facilities and Grounds Department. 

The data was in .CSV format and included four files - one for each substation on UA's main campus. The data was recorded every 15 minutes in power (kW) and listed at the end of the time period. Per Will's instruction, to get the power demand in kWh, one should multiple the power demand by 0.25. In a more verbose form:

\[E[\textrm{kWh}] = P[\textrm{kW}] \cdot 15 \textrm{min} \cdot \frac{1\textrm{h}}{60\textrm{min}}\]

The rest of Section~\ref{assessing-uas-current-electricity-demand} covers importing and "cleaning" the data.

\hypertarget{loading-the-data}{%
\subsubsection{Loading the Data}\label{loading-the-data}}

\begin{tcolorbox}[breakable, size=fbox, boxrule=1pt, pad at break*=1mm,colback=cellbackground, colframe=cellborder]
\prompt{In}{incolor}{2}{\boxspacing}
\begin{Verbatim}[commandchars=\\\{\}]
\PY{n}{df\PYZus{}dict} \PY{o}{=} \PY{p}{\PYZob{}}\PY{p}{\PYZcb{}}
\PY{k}{for} \PY{n}{direction} \PY{o+ow}{in} \PY{p}{[}\PY{l+s+s2}{\PYZdq{}}\PY{l+s+s2}{East}\PY{l+s+s2}{\PYZdq{}}\PY{p}{,} \PY{l+s+s2}{\PYZdq{}}\PY{l+s+s2}{West}\PY{l+s+s2}{\PYZdq{}}\PY{p}{,} \PY{l+s+s2}{\PYZdq{}}\PY{l+s+s2}{North}\PY{l+s+s2}{\PYZdq{}}\PY{p}{,} \PY{l+s+s2}{\PYZdq{}}\PY{l+s+s2}{South}\PY{l+s+s2}{\PYZdq{}}\PY{p}{]}\PY{p}{:}
    \PY{n}{df\PYZus{}dict}\PY{p}{[}\PY{n}{direction}\PY{p}{]} \PY{o}{=} \PY{n}{pd}\PY{o}{.}\PY{n}{read\PYZus{}csv}\PY{p}{(}\PY{l+s+sa}{f}\PY{l+s+s2}{\PYZdq{}}\PY{l+s+s2}{./UA Energy/}\PY{l+s+si}{\PYZob{}}\PY{n}{direction}\PY{l+s+si}{\PYZcb{}}\PY{l+s+s2}{ Sub kW Demand Info \PYZhy{} 2019\PYZhy{}01\PYZhy{}01 \PYZhy{} 2019\PYZhy{}12\PYZhy{}31.csv}\PY{l+s+s2}{\PYZdq{}}\PY{p}{,} 
                                     \PY{n}{skiprows}\PY{o}{=}\PY{l+m+mi}{16}\PY{p}{,}
                                     \PY{n}{parse\PYZus{}dates}\PY{o}{=}\PY{p}{[}\PY{p}{[}\PY{l+s+s1}{\PYZsq{}}\PY{l+s+s1}{Date}\PY{l+s+s1}{\PYZsq{}}\PY{p}{,} \PY{l+s+s1}{\PYZsq{}}\PY{l+s+s1}{Time}\PY{l+s+s1}{\PYZsq{}}\PY{p}{]}\PY{p}{]}\PY{p}{)}
\end{Verbatim}
\end{tcolorbox}

\begin{tcolorbox}[breakable, size=fbox, boxrule=1pt, pad at break*=1mm,colback=cellbackground, colframe=cellborder]
\prompt{In}{incolor}{3}{\boxspacing}
\begin{Verbatim}[commandchars=\\\{\}]
\PY{k}{for} \PY{n}{name}\PY{p}{,} \PY{n}{df} \PY{o+ow}{in} \PY{n}{df\PYZus{}dict}\PY{o}{.}\PY{n}{items}\PY{p}{(}\PY{p}{)}\PY{p}{:}
    \PY{k}{if} \PY{l+s+s1}{\PYZsq{}}\PY{l+s+s1}{TOTAL \PYZhy{} KW\PYZhy{}TOT}\PY{l+s+s1}{\PYZsq{}} \PY{o+ow}{not} \PY{o+ow}{in} \PY{n}{df}\PY{o}{.}\PY{n}{columns}\PY{p}{:}
        \PY{n}{df}\PY{p}{[}\PY{l+s+s1}{\PYZsq{}}\PY{l+s+s1}{TOTAL \PYZhy{} KW\PYZhy{}TOT}\PY{l+s+s1}{\PYZsq{}}\PY{p}{]} \PY{o}{=} \PY{n}{df}\PY{p}{[}\PY{p}{[}\PY{n}{col} \PY{k}{for} \PY{n}{col} \PY{o+ow}{in} \PY{n}{df}\PY{o}{.}\PY{n}{columns} \PY{k}{if} \PY{n}{col} \PY{o+ow}{not} \PY{o+ow}{in} \PY{p}{[}\PY{l+s+s1}{\PYZsq{}}\PY{l+s+s1}{Date\PYZus{}Time}\PY{l+s+s1}{\PYZsq{}}\PY{p}{]}\PY{p}{]}\PY{p}{]}\PY{o}{.}\PY{n}{sum}\PY{p}{(}\PY{n}{axis}\PY{o}{=}\PY{l+m+mi}{1}\PY{p}{)}
    \PY{k}{for} \PY{n}{col} \PY{o+ow}{in} \PY{n}{df}\PY{o}{.}\PY{n}{columns}\PY{p}{:}
        \PY{k}{if} \PY{n}{col} \PY{o+ow}{not} \PY{o+ow}{in} \PY{l+s+s1}{\PYZsq{}}\PY{l+s+s1}{Date\PYZus{}Time}\PY{l+s+s1}{\PYZsq{}}\PY{p}{:}
            \PY{c+c1}{\PYZsh{} Shift the data so that power consumption is associated with the start time}
            \PY{n}{df}\PY{p}{[}\PY{n}{col}\PY{p}{]} \PY{o}{=} \PY{n}{df}\PY{p}{[}\PY{n}{col}\PY{p}{]}\PY{o}{.}\PY{n}{shift}\PY{p}{(}\PY{o}{\PYZhy{}}\PY{l+m+mi}{1}\PY{p}{)}
            \PY{c+c1}{\PYZsh{} drop the last period}
            \PY{n}{df}\PY{o}{.}\PY{n}{drop}\PY{p}{(}\PY{n}{df}\PY{o}{.}\PY{n}{iloc}\PY{p}{[}\PY{o}{\PYZhy{}}\PY{l+m+mi}{1}\PY{p}{]}\PY{o}{.}\PY{n}{name}\PY{p}{,} \PY{n}{inplace}\PY{o}{=}\PY{k+kc}{True}\PY{p}{)}
\end{Verbatim}
\end{tcolorbox}

\hypertarget{creating-the-total-dataframe}{%
\subsubsection{Creating the Total
Dataframe}\label{creating-the-total-dataframe}}

\begin{tcolorbox}[breakable, size=fbox, boxrule=1pt, pad at break*=1mm,colback=cellbackground, colframe=cellborder]
\prompt{In}{incolor}{4}{\boxspacing}
\begin{Verbatim}[commandchars=\\\{\}]
\PY{n}{total\PYZus{}df} \PY{o}{=} \PY{n}{df\PYZus{}dict}\PY{p}{[}\PY{l+s+s1}{\PYZsq{}}\PY{l+s+s1}{East}\PY{l+s+s1}{\PYZsq{}}\PY{p}{]}\PY{p}{[}\PY{p}{[}\PY{l+s+s1}{\PYZsq{}}\PY{l+s+s1}{Date\PYZus{}Time}\PY{l+s+s1}{\PYZsq{}}\PY{p}{,} \PY{l+s+s1}{\PYZsq{}}\PY{l+s+s1}{TOTAL \PYZhy{} KW\PYZhy{}TOT}\PY{l+s+s1}{\PYZsq{}}\PY{p}{]}\PY{p}{]}\PY{o}{.}\PY{n}{copy}\PY{p}{(}\PY{p}{)}
\PY{k}{for} \PY{n}{name}\PY{p}{,} \PY{n}{df} \PY{o+ow}{in} \PY{n}{df\PYZus{}dict}\PY{o}{.}\PY{n}{items}\PY{p}{(}\PY{p}{)}\PY{p}{:}
    \PY{n}{total\PYZus{}df}\PY{p}{[}\PY{l+s+sa}{f}\PY{l+s+s1}{\PYZsq{}}\PY{l+s+si}{\PYZob{}}\PY{n}{name}\PY{l+s+si}{\PYZcb{}}\PY{l+s+s1}{\PYZhy{}TOTAL}\PY{l+s+s1}{\PYZsq{}}\PY{p}{]} \PY{o}{=} \PY{n}{df}\PY{p}{[}\PY{l+s+s1}{\PYZsq{}}\PY{l+s+s1}{TOTAL \PYZhy{} KW\PYZhy{}TOT}\PY{l+s+s1}{\PYZsq{}}\PY{p}{]}\PY{o}{.}\PY{n}{astype}\PY{p}{(}\PY{l+s+s2}{\PYZdq{}}\PY{l+s+s2}{pint[kW]}\PY{l+s+s2}{\PYZdq{}}\PY{p}{)}
    \PY{k}{if} \PY{n}{name} \PY{o+ow}{not} \PY{o+ow}{in} \PY{l+s+s1}{\PYZsq{}}\PY{l+s+s1}{East}\PY{l+s+s1}{\PYZsq{}}\PY{p}{:}
        \PY{n}{total\PYZus{}df}\PY{p}{[}\PY{l+s+s1}{\PYZsq{}}\PY{l+s+s1}{TOTAL \PYZhy{} KW\PYZhy{}TOT}\PY{l+s+s1}{\PYZsq{}}\PY{p}{]} \PY{o}{+}\PY{o}{=} \PY{n}{df}\PY{p}{[}\PY{l+s+s1}{\PYZsq{}}\PY{l+s+s1}{TOTAL \PYZhy{} KW\PYZhy{}TOT}\PY{l+s+s1}{\PYZsq{}}\PY{p}{]}
\PY{n}{total\PYZus{}df}\PY{p}{[}\PY{l+s+s1}{\PYZsq{}}\PY{l+s+s1}{TOTAL \PYZhy{} KW\PYZhy{}TOT}\PY{l+s+s1}{\PYZsq{}}\PY{p}{]} \PY{o}{=} \PY{n}{total\PYZus{}df}\PY{p}{[}\PY{l+s+s1}{\PYZsq{}}\PY{l+s+s1}{TOTAL \PYZhy{} KW\PYZhy{}TOT}\PY{l+s+s1}{\PYZsq{}}\PY{p}{]}\PY{o}{.}\PY{n}{astype}\PY{p}{(}\PY{l+s+s2}{\PYZdq{}}\PY{l+s+s2}{pint[kW]}\PY{l+s+s2}{\PYZdq{}}\PY{p}{)}
\end{Verbatim}
\end{tcolorbox}

\hypertarget{set-the-index-to-a-timezone-aware-datetime}{%
\paragraph{Setting the Index to a Timezone-aware
Datetime}\label{set-the-index-to-a-timezone-aware-datetime}}

\begin{tcolorbox}[breakable, size=fbox, boxrule=1pt, pad at break*=1mm,colback=cellbackground, colframe=cellborder]
\prompt{In}{incolor}{5}{\boxspacing}
\begin{Verbatim}[commandchars=\\\{\}]
\PY{n}{tz} \PY{o}{=} \PY{n}{timezone}\PY{p}{(}\PY{l+s+s1}{\PYZsq{}}\PY{l+s+s1}{US/Central}\PY{l+s+s1}{\PYZsq{}}\PY{p}{,} \PY{p}{)}
\PY{n}{total\PYZus{}df}\PY{p}{[}\PY{l+s+s1}{\PYZsq{}}\PY{l+s+s1}{Date\PYZus{}Time}\PY{l+s+s1}{\PYZsq{}}\PY{p}{]} \PY{o}{=} \PY{n}{total\PYZus{}df}\PY{p}{[}\PY{l+s+s1}{\PYZsq{}}\PY{l+s+s1}{Date\PYZus{}Time}\PY{l+s+s1}{\PYZsq{}}\PY{p}{]}\PY{o}{.}\PY{n}{apply}\PY{p}{(}\PY{n}{tz}\PY{o}{.}\PY{n}{localize}\PY{p}{,} \PY{n}{is\PYZus{}dst}\PY{o}{=}\PY{k+kc}{False}\PY{p}{)}
\PY{n}{total\PYZus{}df}\PY{o}{.}\PY{n}{set\PYZus{}index}\PY{p}{(}\PY{l+s+s1}{\PYZsq{}}\PY{l+s+s1}{Date\PYZus{}Time}\PY{l+s+s1}{\PYZsq{}}\PY{p}{,} \PY{n}{inplace}\PY{o}{=}\PY{k+kc}{True}\PY{p}{,} \PY{n}{drop}\PY{o}{=}\PY{k+kc}{False}\PY{p}{)}
\end{Verbatim}
\end{tcolorbox}

\begin{tcolorbox}[breakable, size=fbox, boxrule=1pt, pad at break*=1mm,colback=cellbackground, colframe=cellborder]
\prompt{In}{incolor}{6}{\boxspacing}
\begin{Verbatim}[commandchars=\\\{\}]
\PY{k}{for} \PY{n}{col} \PY{o+ow}{in} \PY{n}{total\PYZus{}df}\PY{o}{.}\PY{n}{columns}\PY{p}{:}
    \PY{k}{if} \PY{n}{col} \PY{o+ow}{not} \PY{o+ow}{in} \PY{l+s+s1}{\PYZsq{}}\PY{l+s+s1}{Date\PYZus{}Time}\PY{l+s+s1}{\PYZsq{}}\PY{p}{:}
        \PY{n}{total\PYZus{}df}\PY{p}{[}\PY{n}{col}\PY{o}{.}\PY{n}{split}\PY{p}{(}\PY{l+s+s1}{\PYZsq{}}\PY{l+s+s1}{\PYZhy{}}\PY{l+s+s1}{\PYZsq{}}\PY{p}{)}\PY{p}{[}\PY{l+m+mi}{0}\PY{p}{]} \PY{o}{+} \PY{l+s+s2}{\PYZdq{}}\PY{l+s+s2}{ Energy}\PY{l+s+s2}{\PYZdq{}}\PY{p}{]} \PY{o}{=} \PY{n}{total\PYZus{}df}\PY{p}{[}\PY{n}{col}\PY{p}{]} \PY{o}{*} \PY{l+m+mf}{0.25} \PY{o}{*} \PY{n}{ureg}\PY{o}{.}\PY{n}{hour}  \PY{c+c1}{\PYZsh{} in kWh}
\end{Verbatim}
\end{tcolorbox}

\hypertarget{UA-power-and-energy}{%
\subsubsection{Calculating UA's Total Power and Energy Consumption}\label{UA-power-and-energy}}

\begin{tcolorbox}[breakable, size=fbox, boxrule=1pt, pad at break*=1mm,colback=cellbackground, colframe=cellborder]
\prompt{In}{incolor}{7}{\boxspacing}
\begin{Verbatim}[commandchars=\\\{\}]
\PY{n}{total\PYZus{}df}\PY{p}{[}\PY{l+s+s1}{\PYZsq{}}\PY{l+s+s1}{TOTAL \PYZhy{} KW\PYZhy{}TOT}\PY{l+s+s1}{\PYZsq{}}\PY{p}{]}\PY{o}{.}\PY{n}{mean}\PY{p}{(}\PY{p}{)}
\end{Verbatim}
\end{tcolorbox}
 
            
\prompt{Out}{outcolor}{7}{}
    
$27877.04315930694\ \mathrm{kilowatt}$


\begin{tcolorbox}[breakable, size=fbox, boxrule=1pt, pad at break*=1mm,colback=cellbackground, colframe=cellborder]
\prompt{In}{incolor}{8}{\boxspacing}
\begin{Verbatim}[commandchars=\\\{\}]
\PY{n}{total\PYZus{}df}\PY{p}{[}\PY{l+s+s1}{\PYZsq{}}\PY{l+s+s1}{TOTAL  Energy}\PY{l+s+s1}{\PYZsq{}}\PY{p}{]}\PY{o}{.}\PY{n}{sum}\PY{p}{(}\PY{p}{)}\PY{o}{.}\PY{n}{to}\PY{p}{(}\PY{l+s+s1}{\PYZsq{}}\PY{l+s+s1}{GWh}\PY{l+s+s1}{\PYZsq{}}\PY{p}{)}
\end{Verbatim}
\end{tcolorbox}
 
            
\prompt{Out}{outcolor}{8}{}
    
$244.15411325\ \mathrm{gigawatt\_hour}$

% \hypertarget{plotting-the-weekly-power-average-for-2019}
% {%
% \paragraph{Plotting the Weekly Power Average for
% 2019}\label{plotting-the-weekly-power-average-for-2019}}

\rmfamily

In summary, UA used 244 GWh of electricity throughout the year. Assuming that 2019 is representative of the average year at UA, this electricity usage peaks in late September / early October at 36.8 MW. To put that in context, the average house consumes 11,000 kWh per year ~\citep{electricity-use-in-homes}. UA's energy consumption is the equivalent of \textbf{22,181 homes.}

\begin{center}
\adjustimage{max size={0.9\linewidth}{0.9\paperheight}}{Final Project_files/Final Project_21_0.png}
\end{center}
% { \hspace*{\fill} \\}
    
\hypertarget{solar-potential-calculation}{%
\subsection{Solar Potential
Calculation}\label{solar-potential-calculation}}

After calculating UA's power and energy consumption, the analysis turned to analyzing the solar potential of UA's surface parking lots. The first part of this analysis was a simple one done using the formulas and values from this course's textbook, the Fundamentals and Applications of Renewable Energy~\citep{kanouglu2020fundamentals}. Further down, a more detailed analysis was done using the atlite software~\citep{Hofmann_Atlite_A_Light-weight_2021}.

The remaining analysis uses values from the \href{https://us.sunpower.com/sites/default/files/media-library/data-sheets/ds-sunpower-p17-355-1500v-commercial-solar-panels.pdf}{SunPower's datasheet}. SunPower panels were chosen for this analysis as they are one of the most widely adopted brands and had informative documentation on their website. Specifically, the P-Series panels were selected as they are SunPower's utility-scale panels. The most important value from the datasheet is the efficiency of the panels, which for the SunPower P-Series is around 19\%.  

\hypertarget{simplistic-from-the-book}{%
\subsubsection{Simplistic, from the book}\label{simplistic-from-the-book}}

The book provides average solar radiation values (\(\frac{MJ}{m^2 \cdot day}\)) for many cities in the U.S. The nearest city to Tuscaloosa that has values is Birmingham, Alabama, so it is used as an analog. 

\begin{tcolorbox}[breakable, size=fbox, boxrule=1pt, pad at break*=1mm,colback=cellbackground, colframe=cellborder]
\prompt{In}{incolor}{175}{\boxspacing}
\begin{Verbatim}[commandchars=\\\{\}]
\PY{n}{G\PYZus{}solar} \PY{o}{=} \PY{n+nb}{list}\PY{p}{(}\PY{n+nb}{map}\PY{p}{(}\PY{k}{lambda} \PY{n}{x}\PY{p}{:} \PY{n}{x} \PY{o}{*} \PY{n}{ureg}\PY{o}{.}\PY{n}{MJ} \PY{o}{/} \PY{p}{(}\PY{n}{ureg}\PY{o}{.}\PY{n}{m} \PY{o}{*}\PY{o}{*} \PY{l+m+mi}{2} \PY{o}{*} \PY{n}{ureg}\PY{o}{.}\PY{n}{day}\PY{p}{)}\PY{p}{,} \PY{p}{[}\PY{l+m+mf}{9.20}\PY{p}{,} \PY{l+m+mf}{11.92}\PY{p}{,} \PY{l+m+mf}{13.67}\PY{p}{,} \PY{l+m+mf}{19.65}\PY{p}{,} \PY{l+m+mf}{21.58}\PY{p}{,} \PY{l+m+mf}{22.37}\PY{p}{,} \PY{l+m+mf}{21.24}\PY{p}{,} \PY{l+m+mf}{20.21}\PY{p}{,} \PY{l+m+mf}{17.15}\PY{p}{,} \PY{l+m+mf}{14.42}\PY{p}{,} \PY{l+m+mf}{10.22}\PY{p}{,} \PY{l+m+mf}{8.40}\PY{p}{]}\PY{p}{)}\PY{p}{)}
\PY{n}{G\PYZus{}solar} \PY{o}{=} \PY{p}{\PYZob{}}\PY{n}{i} \PY{o}{+} \PY{l+m+mi}{1}\PY{p}{:} \PY{p}{\PYZob{}}\PY{l+s+s1}{\PYZsq{}}\PY{l+s+s1}{G}\PY{l+s+s1}{\PYZsq{}}\PY{p}{:} \PY{n}{g}\PY{p}{,} \PY{l+s+s1}{\PYZsq{}}\PY{l+s+s1}{D}\PY{l+s+s1}{\PYZsq{}}\PY{p}{:} \PY{p}{(}\PY{l+m+mi}{30} \PY{k}{if} \PY{n}{i} \PY{o}{\PYZpc{}} \PY{l+m+mi}{2} \PY{k}{else} \PY{l+m+mi}{31}\PY{p}{)} \PY{o}{*} \PY{n}{ureg}\PY{o}{.}\PY{n}{day}\PY{p}{\PYZcb{}} \PY{k}{for} \PY{n}{i}\PY{p}{,} \PY{n}{g} \PY{o+ow}{in} \PY{n+nb}{enumerate}\PY{p}{(}\PY{n}{G\PYZus{}solar}\PY{p}{)}\PY{p}{\PYZcb{}}
\PY{c+c1}{\PYZsh{} Manually Correcting Feb. Days}
\PY{n}{G\PYZus{}solar}\PY{p}{[}\PY{l+m+mi}{2}\PY{p}{]}\PY{p}{[}\PY{l+s+s1}{\PYZsq{}}\PY{l+s+s1}{D}\PY{l+s+s1}{\PYZsq{}}\PY{p}{]} \PY{o}{=} \PY{l+m+mi}{28} \PY{o}{*} \PY{n}{ureg}\PY{o}{.}\PY{n}{day}
\end{Verbatim}
\end{tcolorbox}

\rmfamily
Using Google Maps measure tool, I calculated the surface area of all the considered parking lots and saved it to a CSV, along with the GPS coordinates to each of the parking lots centers.

\begin{tcolorbox}[breakable, size=fbox, boxrule=1pt, pad at break*=1mm,colback=cellbackground, colframe=cellborder]
\prompt{In}{incolor}{12}{\boxspacing}
\begin{Verbatim}[commandchars=\\\{\}]
\PY{n}{parking\PYZus{}lots} \PY{o}{=} \PY{n}{pd}\PY{o}{.}\PY{n}{read\PYZus{}csv}\PY{p}{(}\PY{l+s+s2}{\PYZdq{}}\PY{l+s+s2}{ParkingLotArea.csv}\PY{l+s+s2}{\PYZdq{}}\PY{p}{)}
\PY{n}{parking\PYZus{}lots}\PY{o}{.}\PY{n}{head}\PY{p}{(}\PY{p}{)}
\end{Verbatim}
\end{tcolorbox}

\begin{tcolorbox}[breakable, size=fbox, boxrule=1pt, pad at break*=1mm,colback=cellbackground, colframe=cellborder]
\prompt{In}{incolor}{13}{\boxspacing}
\begin{Verbatim}[commandchars=\\\{\}]
\PY{n}{parking\PYZus{}lots}\PY{p}{[}\PY{l+s+s1}{\PYZsq{}}\PY{l+s+s1}{Area (ft)}\PY{l+s+s1}{\PYZsq{}}\PY{p}{]} \PY{o}{=} \PY{n}{parking\PYZus{}lots}\PY{p}{[}\PY{l+s+s1}{\PYZsq{}}\PY{l+s+s1}{Area (ft)}\PY{l+s+s1}{\PYZsq{}}\PY{p}{]}\PY{o}{.}\PY{n}{astype}\PY{p}{(}\PY{l+s+s2}{\PYZdq{}}\PY{l+s+s2}{pint[ft\PYZca{}2]}\PY{l+s+s2}{\PYZdq{}}\PY{p}{)}
\end{Verbatim}
\end{tcolorbox}

\hypertarget{calculating-the-theoretical-energy-per-day}{%
\paragraph{Calculating the Theoretical Energy per
Day}\label{calculating-the-theoretical-energy-per-day}}

Using the 19\% efficiency of the SunPower P-series panels (\(N_{pv}\)), the power in terms of \(\frac{MJ}{day}\) can be calculated for every month of the year. The equation is as follows:

\[P_{\textrm{parking lot}_{\textrm{month}}} = A_{\textrm{parking lot}} \cdot G_{\textrm{month}} \cdot N_{pv}\]

The loop below calculates the power for each parking lot and for each month in the year.

\begin{tcolorbox}[breakable, size=fbox, boxrule=1pt, pad at break*=1mm,colback=cellbackground, colframe=cellborder]
\prompt{In}{incolor}{176}{\boxspacing}
\begin{Verbatim}[commandchars=\\\{\}]
\PY{n}{N\PYZus{}pv} \PY{o}{=} \PY{l+m+mf}{0.19} \PY{c+c1}{\PYZsh{} the efficiency of SunPower P\PYZhy{}series }
\end{Verbatim}
\end{tcolorbox}

\begin{tcolorbox}[breakable, size=fbox, boxrule=1pt, pad at break*=1mm,colback=cellbackground, colframe=cellborder]
\prompt{In}{incolor}{177}{\boxspacing}
\begin{Verbatim}[commandchars=\\\{\}]
\PY{k}{for} \PY{n}{month}\PY{p}{,} \PY{n}{inner\PYZus{}dict} \PY{o+ow}{in} \PY{n}{G\PYZus{}solar}\PY{o}{.}\PY{n}{items}\PY{p}{(}\PY{p}{)}\PY{p}{:}
\PY{n}{parking\PYZus{}lots}\PY{p}{[}\PY{l+s+sa}{f}\PY{l+s+s1}{\PYZsq{}}\PY{l+s+s1}{P\PYZus{}}\PY{l+s+si}{\PYZob{}}\PY{n}{month}\PY{l+s+si}{\PYZcb{}}\PY{l+s+s1}{\PYZsq{}}\PY{p}{]} \PY{o}{=} \PY{p}{(}\PY{n}{parking\PYZus{}lots}\PY{p}{[}\PY{l+s+s1}{\PYZsq{}}\PY{l+s+s1}{Area (ft)}\PY{l+s+s1}{\PYZsq{}}\PY{p}{]} \PY{o}{*} \PY{n}{N\PYZus{}pv} \PY{o}{*} \PY{n}{inner\PYZus{}dict}\PY{p}{[}\PY{l+s+s1}{\PYZsq{}}\PY{l+s+s1}{G}\PY{l+s+s1}{\PYZsq{}}\PY{p}{]}\PY{p}{)}\PY{o}{.}\PY{n}{pint}\PY{o}{.}\PY{n}{to}\PY{p}{(}\PY{l+s+s2}{\PYZdq{}}\PY{l+s+s2}{MJ / day}\PY{l+s+s2}{\PYZdq{}}\PY{p}{)}
\end{Verbatim}
\end{tcolorbox}

\hypertarget{average-daily-per-lot-and-total}{%
\paragraph{Average Daily per Lot and
Total}\label{average-daily-per-lot-and-total}}

Using the monthly power calculated in Section~\ref{calculating-the-theoretical-energy-per-day}, the yearly average values can be found for each parking lot by taking the weighted average of each months' average power, where the weights are the days in the month.

\begin{tcolorbox}[breakable, size=fbox, boxrule=1pt, pad at break*=1mm,colback=cellbackground, colframe=cellborder]
\prompt{In}{incolor}{16}{\boxspacing}
\begin{Verbatim}[commandchars=\\\{\}]
\PY{n}{parking\PYZus{}lots}\PY{p}{[}\PY{l+s+s1}{\PYZsq{}}\PY{l+s+s1}{P\PYZus{}avg}\PY{l+s+s1}{\PYZsq{}}\PY{p}{]} \PY{o}{=} \PY{l+m+mi}{0} 
\PY{n}{parking\PYZus{}lots}\PY{p}{[}\PY{l+s+s1}{\PYZsq{}}\PY{l+s+s1}{P\PYZus{}avg}\PY{l+s+s1}{\PYZsq{}}\PY{p}{]} \PY{o}{=} \PY{n}{parking\PYZus{}lots}\PY{p}{[}\PY{l+s+s1}{\PYZsq{}}\PY{l+s+s1}{P\PYZus{}avg}\PY{l+s+s1}{\PYZsq{}}\PY{p}{]}\PY{o}{.}\PY{n}{astype}\PY{p}{(}\PY{l+s+s1}{\PYZsq{}}\PY{l+s+s1}{pint[MJ / day]}\PY{l+s+s1}{\PYZsq{}}\PY{p}{)} 

\PY{k}{for} \PY{n}{month}\PY{p}{,} \PY{n}{inner\PYZus{}dict} \PY{o+ow}{in} \PY{n}{G\PYZus{}solar}\PY{o}{.}\PY{n}{items}\PY{p}{(}\PY{p}{)}\PY{p}{:}
    \PY{n}{parking\PYZus{}lots}\PY{p}{[}\PY{l+s+s1}{\PYZsq{}}\PY{l+s+s1}{P\PYZus{}avg}\PY{l+s+s1}{\PYZsq{}}\PY{p}{]} \PY{o}{+}\PY{o}{=} \PY{n}{parking\PYZus{}lots}\PY{p}{[}\PY{l+s+sa}{f}\PY{l+s+s1}{\PYZsq{}}\PY{l+s+s1}{P\PYZus{}}\PY{l+s+si}{\PYZob{}}\PY{n}{month}\PY{l+s+si}{\PYZcb{}}\PY{l+s+s1}{\PYZsq{}}\PY{p}{]} \PY{o}{*} \PY{n}{inner\PYZus{}dict}\PY{p}{[}\PY{l+s+s1}{\PYZsq{}}\PY{l+s+s1}{D}\PY{l+s+s1}{\PYZsq{}}\PY{p}{]} \PY{o}{/} \PY{p}{(}\PY{l+m+mi}{365} \PY{o}{*} \PY{n}{ureg}\PY{o}{.}\PY{n}{day}\PY{p}{)}
\end{Verbatim}
\end{tcolorbox}

\begin{tcolorbox}[breakable, size=fbox, boxrule=1pt, pad at break*=1mm,colback=cellbackground, colframe=cellborder]
\prompt{In}{incolor}{17}{\boxspacing}
\begin{Verbatim}[commandchars=\\\{\}]
\PY{n}{parking\PYZus{}lots}\PY{p}{[}\PY{l+s+s1}{\PYZsq{}}\PY{l+s+s1}{P\PYZus{}total}\PY{l+s+s1}{\PYZsq{}}\PY{p}{]} \PY{o}{=} \PY{p}{(}\PY{n}{parking\PYZus{}lots}\PY{p}{[}\PY{l+s+s1}{\PYZsq{}}\PY{l+s+s1}{P\PYZus{}avg}\PY{l+s+s1}{\PYZsq{}}\PY{p}{]} \PY{o}{*} \PY{l+m+mi}{365} \PY{o}{*} \PY{n}{ureg}\PY{o}{.}\PY{n}{day}\PY{p}{)}
\end{Verbatim}
\end{tcolorbox}

\hypertarget{plotting-the-installed-capacity-vs.-monthly-power-consumption}{%
\paragraph{Plotting the Calculated Solar Power vs. Monthly Power
Consumption}\label{plotting-the-installed-capacity-vs.-monthly-power-consumption}}

In the figure below, the simple solar power calculations from the book are plotted against UA's power demand throughout the year. 

\begin{center}
\adjustimage{max size={0.9\linewidth}{0.9\paperheight}}{Final Project_files/Final Project_45_0.png}
\end{center}
{ \hspace*{\fill} \\}

\hypertarget{complicated-capacity-estimate}{%
\subsubsection{Complicated Capacity
Estimate}\label{complicated-capacity-estimate}}

In Section~\ref{simplistic-from-the-book}, the average daily sun power that reaches the parking lot surfaces was calculated, which ignores several details. In this section, the atlite Python package is utilized to estimate the actual available solar power with more accuracy~\citet{Hofmann_Atlite_A_Light-weight_2021}.

\hypertarget{calculating-the-installed-capacity-using-the-available-area-and-sun-powers-datasheet}{%
\paragraph{Calculating the installed capacity, using the available area
and Sun Power's
Datasheet}\label{calculating-the-installed-capacity-using-the-available-area-and-sun-powers-datasheet}}
\rmfamily

One of the inputs to the atlite solar calculations is the \emph{installed capacity}, which is different than average power used in Section~\ref{simplistic-from-the-book}. The total solar capacity is calculated as

\[\textrm{Capacity}_{\textrm{solar}} = N_{panels} \cdot \dot{W}_{panel} * N_{\textrm{ac/dc}}\]

where \(N_{panels}\) is the number of panels that can fit on the parking
lots and $ \dot{W}_{panel}$ is the nominal max power of the panel, which comes from the datasheet. For the P-series panel, the nominal power is 335 Watts. Because solar panels generate DC power and UA uses standard AC power, an inverter would per required. $N_{\textrm{ac/dc}}$ represents the inverter efficiency and is assumed to be 90\%. 

The formula that I used to find the number of panels for each parking lot is below:

\[N_{panels} = \textrm{floor}\left(\frac{A_{\textrm{parking lot}}}{A_{\textrm{solar panel}}} \cdot N_{\textrm{area}}\right)\]

where \(N_{\textrm{area}}\) is a scaling factor \(< 1\) to account for
the fact that not 100\% of the parking lot surface area can likely be
utilized by solar panels, as there will be transformer hardware etc.

\begin{tcolorbox}[breakable, size=fbox, boxrule=1pt, pad at break*=1mm,colback=cellbackground, colframe=cellborder]
\prompt{In}{incolor}{181}{\boxspacing}
\begin{Verbatim}[commandchars=\\\{\}]
\PY{n}{P\PYZus{}pv} \PY{o}{=} \PY{l+m+mi}{355} \PY{o}{*} \PY{n}{ureg}\PY{o}{.}\PY{n}{watt}
\PY{n}{A\PYZus{}pv} \PY{o}{=} \PY{l+m+mf}{81.4} \PY{o}{*} \PY{l+m+mf}{39.3} \PY{o}{*} \PY{n}{ureg}\PY{o}{.}\PY{n}{inch} \PY{o}{*}\PY{o}{*} \PY{l+m+mi}{2}
\PY{n}{N\PYZus{}ac\PYZus{}dc} \PY{o}{=} \PY{l+m+mf}{.9}
\PY{n}{N\PYZus{}area} \PY{o}{=} \PY{l+m+mf}{0.95}  \PY{c+c1}{\PYZsh{} estimating that not all surface area will be covered by panels due to geometry constraints etc.}
\end{Verbatim}
\end{tcolorbox}

\hypertarget{determining-the-number-of-sunpower-p17-panels-that-can-fit-on-each-surface}{%
\paragraph{Determining the number of SunPower P17 Panels that can fit
on each
surface}\label{determining-the-number-of-sunpower-p17-panels-that-can-fit-on-each-surface}}

\begin{tcolorbox}[breakable, size=fbox, boxrule=1pt, pad at break*=1mm,colback=cellbackground, colframe=cellborder]
\prompt{In}{incolor}{182}{\boxspacing}
\begin{Verbatim}[commandchars=\\\{\}]
\PY{n}{parking\PYZus{}lots}\PY{p}{[}\PY{l+s+s1}{\PYZsq{}}\PY{l+s+s1}{\PYZsh{} of Panels}\PY{l+s+s1}{\PYZsq{}}\PY{p}{]} \PY{o}{=} \PY{p}{(}\PY{n}{parking\PYZus{}lots}\PY{p}{[}\PY{l+s+s1}{\PYZsq{}}\PY{l+s+s1}{Area (ft)}\PY{l+s+s1}{\PYZsq{}}\PY{p}{]} \PY{o}{/} \PY{n}{A\PYZus{}pv} \PY{o}{*} \PY{n}{N\PYZus{}area}\PY{p}{)}\PY{o}{.}\PY{n}{pint}\PY{o}{.}\PY{n}{to}\PY{p}{(}\PY{l+s+s1}{\PYZsq{}}\PY{l+s+s1}{dimensionless}\PY{l+s+s1}{\PYZsq{}}\PY{p}{)}
\PY{n}{parking\PYZus{}lots}\PY{p}{[}\PY{l+s+s1}{\PYZsq{}}\PY{l+s+s1}{\PYZsh{} of Panels}\PY{l+s+s1}{\PYZsq{}}\PY{p}{]} \PY{o}{=} \PY{n}{np}\PY{o}{.}\PY{n}{floor}\PY{p}{(}\PY{n}{parking\PYZus{}lots}\PY{p}{[}\PY{l+s+s1}{\PYZsq{}}\PY{l+s+s1}{\PYZsh{} of Panels}\PY{l+s+s1}{\PYZsq{}}\PY{p}{]}\PY{p}{)}
\PY{n}{parking\PYZus{}lots}\PY{p}{[}\PY{l+s+s1}{\PYZsq{}}\PY{l+s+s1}{\PYZsh{} of Panels}\PY{l+s+s1}{\PYZsq{}}\PY{p}{]}
\end{Verbatim}
\end{tcolorbox}

\begin{tcolorbox}[breakable, size=fbox, boxrule=.5pt, pad at break*=1mm, opacityfill=0]
\prompt{Out}{outcolor}{182}{\boxspacing}
\begin{Verbatim}[commandchars=\\\{\}]
Lot                     # of Panels
NE of NE Commuter       10038
Cent. of NE Commuter    8833
S of NE Commuter        4820
Frat of NE Commuter     12121
Cyber Hall Parking      4189
Yellow Zones            4910
Bryce Hospital          5730
West Commuter           7924
SE Commuter Coleman     9089
SE Commuter Law         5600
\end{Verbatim}
\end{tcolorbox}
        
\begin{tcolorbox}[breakable, size=fbox, boxrule=1pt, pad at break*=1mm,colback=cellbackground, colframe=cellborder]
\prompt{In}{incolor}{183}{\boxspacing}
\begin{Verbatim}[commandchars=\\\{\}]
\PY{n}{parking\PYZus{}lots}\PY{p}{[}\PY{l+s+s1}{\PYZsq{}}\PY{l+s+s1}{PV Capacity}\PY{l+s+s1}{\PYZsq{}}\PY{p}{]} \PY{o}{=} \PY{n}{parking\PYZus{}lots}\PY{p}{[}\PY{l+s+s1}{\PYZsq{}}\PY{l+s+s1}{\PYZsh{} of Panels}\PY{l+s+s1}{\PYZsq{}}\PY{p}{]} \PY{o}{*} \PY{n}{P\PYZus{}pv} \PY{o}{*} \PY{n}{N\PYZus{}ac\PYZus{}dc}
\PY{n}{parking\PYZus{}lots}\PY{p}{[}\PY{l+s+s1}{\PYZsq{}}\PY{l+s+s1}{PV Capacity}\PY{l+s+s1}{\PYZsq{}}\PY{p}{]} \PY{o}{=} \PY{n}{parking\PYZus{}lots}\PY{p}{[}\PY{l+s+s1}{\PYZsq{}}\PY{l+s+s1}{PV Capacity}\PY{l+s+s1}{\PYZsq{}}\PY{p}{]}\PY{o}{.}\PY{n}{astype}\PY{p}{(}\PY{l+s+s1}{\PYZsq{}}\PY{l+s+s1}{pint[W]}\PY{l+s+s1}{\PYZsq{}}\PY{p}{)}
\PY{n}{parking\PYZus{}lots}\PY{p}{[}\PY{l+s+s1}{\PYZsq{}}\PY{l+s+s1}{PV Capacity}\PY{l+s+s1}{\PYZsq{}}\PY{p}{]} \PY{o}{=} \PY{n}{parking\PYZus{}lots}\PY{p}{[}\PY{l+s+s1}{\PYZsq{}}\PY{l+s+s1}{PV Capacity}\PY{l+s+s1}{\PYZsq{}}\PY{p}{]}\PY{o}{.}\PY{n}{pint}\PY{o}{.}\PY{n}{to}\PY{p}{(}\PY{l+s+s1}{\PYZsq{}}\PY{l+s+s1}{MW}\PY{l+s+s1}{\PYZsq{}}\PY{p}{)}
\PY{c+c1}{\PYZsh{} parking\PYZus{}lots[\PYZsq{}PV Capacity\PYZsq{}]}
\end{Verbatim}
\end{tcolorbox}

\begin{tcolorbox}[breakable, size=fbox, boxrule=1pt, pad at break*=1mm,colback=cellbackground, colframe=cellborder]
\prompt{In}{incolor}{184}{\boxspacing}
\begin{Verbatim}[commandchars=\\\{\}]
\PY{n}{parking\PYZus{}lots}\PY{p}{[}\PY{l+s+s1}{\PYZsq{}}\PY{l+s+s1}{PV Capacity}\PY{l+s+s1}{\PYZsq{}}\PY{p}{]}\PY{o}{.}\PY{n}{sum}\PY{p}{(}\PY{p}{)}
\end{Verbatim}
\end{tcolorbox}
 
\prompt{Out}{outcolor}{184}{}
    
$23.404653\ \mathrm{megawatt}$

\rmfamily
Using the constants above, the total installed capacity is 23.4 MW, which was cross-referenced by visiting \href{https://pvwatts.nrel.gov/pvwatts.php}{NREL's PVWatts website}~\citep{pvwatts-calculator}. Both PVWatts and my calculations above agree.

\hypertarget{using-the-era5-dataset}{%
\paragraph{Using the ERA5 Dataset}\label{using-the-era5-dataset}}

The atlite package relies on the ERA5 dataset for it's solar calculations. The data has been collected since 1979 and provides hourly estimates of a large number of atmospheric, land and oceanic climate variables~\citep{hersbach2018era5}. For this project, the relevant dataset is solar radiation at the earth's surface.

Data from 2015 was used in the calculations below as it was the most recent year with data for the state of Alabama. This is an assumption that would matter more the finer the time resolution of the analysis is, but for the scope of this project I felt that it was an okay assumption.

\begin{tcolorbox}[breakable, size=fbox, boxrule=1pt, pad at break*=1mm,colback=cellbackground, colframe=cellborder]
\prompt{In}{incolor}{186}{\boxspacing}
\begin{Verbatim}[commandchars=\\\{\}]
\PY{n}{bl} \PY{o}{=} \PY{l+m+mf}{32.988320041698074}\PY{p}{,} \PY{o}{\PYZhy{}}\PY{l+m+mf}{87.79145924201974}
\PY{n}{tr} \PY{o}{=} \PY{l+m+mf}{33.473235950702524}\PY{p}{,} \PY{o}{\PYZhy{}}\PY{l+m+mf}{87.32768306322347}

\PY{n}{cutout} \PY{o}{=} \PY{n}{atlite}\PY{o}{.}\PY{n}{Cutout}\PY{p}{(}\PY{n}{path}\PY{o}{=}\PY{l+s+s2}{\PYZdq{}}\PY{l+s+s2}{Alabama.nc}\PY{l+s+s2}{\PYZdq{}}\PY{p}{,}
                       \PY{n}{module}\PY{o}{=}\PY{l+s+s2}{\PYZdq{}}\PY{l+s+s2}{era5}\PY{l+s+s2}{\PYZdq{}}\PY{p}{,}
                       \PY{n}{x}\PY{o}{=}\PY{n+nb}{slice}\PY{p}{(}\PY{n}{bl}\PY{p}{[}\PY{l+m+mi}{1}\PY{p}{]}\PY{p}{,} \PY{n}{tr}\PY{p}{[}\PY{l+m+mi}{1}\PY{p}{]}\PY{p}{)}\PY{p}{,}
                       \PY{n}{y}\PY{o}{=}\PY{n+nb}{slice}\PY{p}{(}\PY{n}{bl}\PY{p}{[}\PY{l+m+mi}{0}\PY{p}{]}\PY{p}{,} \PY{n}{tr}\PY{p}{[}\PY{l+m+mi}{0}\PY{p}{]}\PY{p}{)}\PY{p}{,}
                       \PY{n}{time}\PY{o}{=}\PY{l+s+s2}{\PYZdq{}}\PY{l+s+s2}{2015}\PY{l+s+s2}{\PYZdq{}}  \PY{c+c1}{\PYZsh{} \PYZdq{}2018\PYZhy{}12\PYZdq{})}
                       \PY{p}{)}
\end{Verbatim}
\end{tcolorbox}

\hypertarget{downloading-era5-data-for-the-specified-cutout}{%
\paragraph{Downloading ERA5 Data for the specified
cutout}\label{downloading-era5-data-for-the-specified-cutout}}

\begin{tcolorbox}[breakable, size=fbox, boxrule=1pt, pad at break*=1mm,colback=cellbackground, colframe=cellborder]
\prompt{In}{incolor}{24}{\boxspacing}
\begin{Verbatim}[commandchars=\\\{\}]
\PY{n}{cutout}\PY{o}{.}\PY{n}{prepare}\PY{p}{(}\PY{p}{)}
\end{Verbatim}
\end{tcolorbox}

\hypertarget{finding-the-cutout-cell-nearest-to-the-ua-parking-lots}{%
\paragraph{Finding the Cutout Cell Nearest to the UA Parking
Lots}\label{finding-the-cutout-cell-nearest-to-the-ua-parking-lots}}

The ERA5 data is partitioned into "cut-outs" that are larger than the parking lots themselves. To account for this, all of the parking lots considered for this project where assigned to the nearest ERA5 cell below.

\begin{tcolorbox}[breakable, size=fbox, boxrule=1pt, pad at break*=1mm,colback=cellbackground, colframe=cellborder]
\prompt{In}{incolor}{187}{\boxspacing}
\begin{Verbatim}[commandchars=\\\{\}]
\PY{n}{cells} \PY{o}{=} \PY{n}{gpd}\PY{o}{.}\PY{n}{GeoDataFrame}\PY{p}{(}\PY{p}{\PYZob{}}\PY{l+s+s1}{\PYZsq{}}\PY{l+s+s1}{geometry}\PY{l+s+s1}{\PYZsq{}}\PY{p}{:} \PY{n}{cutout}\PY{o}{.}\PY{n}{grid\PYZus{}cells}\PY{p}{,}
  \PY{l+s+s1}{\PYZsq{}}\PY{l+s+s1}{lon}\PY{l+s+s1}{\PYZsq{}}\PY{p}{:} \PY{n}{cutout}\PY{o}{.}\PY{n}{grid\PYZus{}coordinates}\PY{p}{(}\PY{p}{)}\PY{p}{[}\PY{p}{:}\PY{p}{,}\PY{l+m+mi}{0}\PY{p}{]}\PY{p}{,}
  \PY{l+s+s1}{\PYZsq{}}\PY{l+s+s1}{lat}\PY{l+s+s1}{\PYZsq{}}\PY{p}{:} \PY{n}{cutout}\PY{o}{.}\PY{n}{grid\PYZus{}coordinates}\PY{p}{(}\PY{p}{)}\PY{p}{[}\PY{p}{:}\PY{p}{,}\PY{l+m+mi}{1}\PY{p}{]}\PY{p}{\PYZcb{}}\PY{p}{)}
\end{Verbatim}
\end{tcolorbox}

    \begin{tcolorbox}[breakable, size=fbox, boxrule=1pt, pad at break*=1mm,colback=cellbackground, colframe=cellborder]
\prompt{In}{incolor}{188}{\boxspacing}
\begin{Verbatim}[commandchars=\\\{\}]
\PY{n}{nearest\PYZus{}cell} \PY{o}{=} \PY{n}{cutout}\PY{o}{.}\PY{n}{data}\PY{o}{.}\PY{n}{sel}\PY{p}{(}\PY{p}{\PYZob{}}\PY{l+s+s1}{\PYZsq{}}\PY{l+s+s1}{x}\PY{l+s+s1}{\PYZsq{}}\PY{p}{:} \PY{n}{parking\PYZus{}lots}\PY{o}{.}\PY{n}{Lon}\PY{o}{.}\PY{n}{values}\PY{p}{,}
                                \PY{l+s+s1}{\PYZsq{}}\PY{l+s+s1}{y}\PY{l+s+s1}{\PYZsq{}}\PY{p}{:} \PY{n}{parking\PYZus{}lots}\PY{o}{.}\PY{n}{Lat}\PY{o}{.}\PY{n}{values}\PY{p}{\PYZcb{}}\PY{p}{,}
                               \PY{l+s+s1}{\PYZsq{}}\PY{l+s+s1}{nearest}\PY{l+s+s1}{\PYZsq{}}\PY{p}{)}\PY{o}{.}\PY{n}{coords}
\end{Verbatim}
\end{tcolorbox}

    \begin{tcolorbox}[breakable, size=fbox, boxrule=1pt, pad at break*=1mm,colback=cellbackground, colframe=cellborder]
\prompt{In}{incolor}{189}{\boxspacing}
\begin{Verbatim}[commandchars=\\\{\}]
\PY{c+c1}{\PYZsh{} Map capacities to closest cell coordinate}
\PY{n}{parking\PYZus{}lots}\PY{p}{[}\PY{l+s+s1}{\PYZsq{}}\PY{l+s+s1}{lon}\PY{l+s+s1}{\PYZsq{}}\PY{p}{]} \PY{o}{=} \PY{n}{nearest\PYZus{}cell}\PY{o}{.}\PY{n}{get}\PY{p}{(}\PY{l+s+s1}{\PYZsq{}}\PY{l+s+s1}{lon}\PY{l+s+s1}{\PYZsq{}}\PY{p}{)}\PY{o}{.}\PY{n}{values}
\PY{n}{parking\PYZus{}lots}\PY{p}{[}\PY{l+s+s1}{\PYZsq{}}\PY{l+s+s1}{lat}\PY{l+s+s1}{\PYZsq{}}\PY{p}{]} \PY{o}{=} \PY{n}{nearest\PYZus{}cell}\PY{o}{.}\PY{n}{get}\PY{p}{(}\PY{l+s+s1}{\PYZsq{}}\PY{l+s+s1}{lat}\PY{l+s+s1}{\PYZsq{}}\PY{p}{)}\PY{o}{.}\PY{n}{values}
\end{Verbatim}
\end{tcolorbox}

    \begin{tcolorbox}[breakable, size=fbox, boxrule=1pt, pad at break*=1mm,colback=cellbackground, colframe=cellborder]
\prompt{In}{incolor}{190}{\boxspacing}
\begin{Verbatim}[commandchars=\\\{\}]
\PY{n}{parking\PYZus{}lots}\PY{p}{[}\PY{l+s+s1}{\PYZsq{}}\PY{l+s+s1}{PV Capacity}\PY{l+s+s1}{\PYZsq{}}\PY{p}{]} \PY{o}{=} \PY{n}{parking\PYZus{}lots}\PY{p}{[}\PY{l+s+s1}{\PYZsq{}}\PY{l+s+s1}{PV Capacity}\PY{l+s+s1}{\PYZsq{}}\PY{p}{]}\PY{o}{.}\PY{n}{pint}\PY{o}{.}\PY{n}{magnitude}
\end{Verbatim}
\end{tcolorbox}

    \begin{tcolorbox}[breakable, size=fbox, boxrule=1pt, pad at break*=1mm,colback=cellbackground, colframe=cellborder]
\prompt{In}{incolor}{29}{\boxspacing}
\begin{Verbatim}[commandchars=\\\{\}]
\PY{n}{new\PYZus{}data} \PY{o}{=} \PY{n}{parking\PYZus{}lots}\PY{o}{.}\PY{n}{merge}\PY{p}{(}\PY{n}{cells}\PY{p}{,} \PY{n}{how}\PY{o}{=}\PY{l+s+s1}{\PYZsq{}}\PY{l+s+s1}{inner}\PY{l+s+s1}{\PYZsq{}}\PY{p}{)}

\PY{c+c1}{\PYZsh{} Sum capacities for each grid cell (lat, lon)}
\PY{c+c1}{\PYZsh{} then: restore lat lon as columns}
\PY{c+c1}{\PYZsh{} then: rename and reindex to match cutout coordinates}
\PY{n}{new\PYZus{}data} \PY{o}{=} \PY{n}{new\PYZus{}data}\PY{o}{.}\PY{n}{groupby}\PY{p}{(}\PY{p}{[}\PY{l+s+s1}{\PYZsq{}}\PY{l+s+s1}{lon}\PY{l+s+s1}{\PYZsq{}}\PY{p}{,}\PY{l+s+s1}{\PYZsq{}}\PY{l+s+s1}{lat}\PY{l+s+s1}{\PYZsq{}}\PY{p}{]}\PY{p}{)}\PY{o}{.}\PY{n}{sum}\PY{p}{(}\PY{p}{)}

\PY{n}{layout} \PY{o}{=} \PY{n}{new\PYZus{}data}\PY{o}{.}\PY{n}{reset\PYZus{}index}\PY{p}{(}\PY{p}{)}\PY{o}{.}\PY{n}{rename}\PY{p}{(}\PY{n}{columns}\PY{o}{=}\PY{p}{\PYZob{}}\PY{l+s+s1}{\PYZsq{}}\PY{l+s+s1}{lat}\PY{l+s+s1}{\PYZsq{}}\PY{p}{:}\PY{l+s+s1}{\PYZsq{}}\PY{l+s+s1}{y}\PY{l+s+s1}{\PYZsq{}}\PY{p}{,}\PY{l+s+s1}{\PYZsq{}}\PY{l+s+s1}{lon}\PY{l+s+s1}{\PYZsq{}}\PY{p}{:}\PY{l+s+s1}{\PYZsq{}}\PY{l+s+s1}{x}\PY{l+s+s1}{\PYZsq{}}\PY{p}{\PYZcb{}}\PY{p}{)}\PYZbs{}
                    \PY{o}{.}\PY{n}{set\PYZus{}index}\PY{p}{(}\PY{p}{[}\PY{l+s+s1}{\PYZsq{}}\PY{l+s+s1}{y}\PY{l+s+s1}{\PYZsq{}}\PY{p}{,}\PY{l+s+s1}{\PYZsq{}}\PY{l+s+s1}{x}\PY{l+s+s1}{\PYZsq{}}\PY{p}{]}\PY{p}{)}\PY{p}{[}\PY{l+s+s1}{\PYZsq{}}\PY{l+s+s1}{PV Capacity}\PY{l+s+s1}{\PYZsq{}}\PY{p}{]}\PYZbs{}
                    \PY{o}{.}\PY{n}{to\PYZus{}xarray}\PY{p}{(}\PY{p}{)}\PY{o}{.}\PY{n}{reindex\PYZus{}like}\PY{p}{(}\PY{n}{cutout}\PY{o}{.}\PY{n}{data}\PY{p}{)}

\PY{n}{layout} \PY{o}{=} \PY{n}{layout}\PY{o}{.}\PY{n}{fillna}\PY{p}{(}\PY{l+m+mf}{.0}\PY{p}{)}\PY{o}{.}\PY{n}{rename}\PY{p}{(}\PY{l+s+s1}{\PYZsq{}}\PY{l+s+s1}{Installed Capacity [MW]}\PY{l+s+s1}{\PYZsq{}}\PY{p}{)}
\end{Verbatim}
\end{tcolorbox}

\rmfamily
The atlite package allows the user to configure the orientation of the solar panels. Because the class did not explicitly cover solar panel orientation calculations, I chose to let atlite calculate the optimal orientation given the latitude of Tuscaloosa, Alabama. 

\begin{tcolorbox}[breakable, size=fbox, boxrule=1pt, pad at break*=1mm,colback=cellbackground, colframe=cellborder]
\prompt{In}{incolor}{30}{\boxspacing}
\begin{Verbatim}[commandchars=\\\{\}]
\PY{n}{pv} \PY{o}{=} \PY{n}{cutout}\PY{o}{.}\PY{n}{pv}\PY{p}{(}\PY{n}{panel}\PY{o}{=}\PY{l+s+s2}{\PYZdq{}}\PY{l+s+s2}{CSi}\PY{l+s+s2}{\PYZdq{}}\PY{p}{,} \PY{n}{orientation}\PY{o}{=}\PY{l+s+s1}{\PYZsq{}}\PY{l+s+s1}{latitude\PYZus{}optimal}\PY{l+s+s1}{\PYZsq{}}\PY{p}{,} \PY{n}{layout}\PY{o}{=}\PY{n}{layout}\PY{p}{)}
\end{Verbatim}
\end{tcolorbox}

\hypertarget{put-the-data-into-a-dataframe-and-fixing-the-timezone}{%
\paragraph{Put the data into a DataFrame and fixing the
timezone}\label{put-the-data-into-a-dataframe-and-fixing-the-timezone}}

\begin{tcolorbox}[breakable, size=fbox, boxrule=1pt, pad at break*=1mm,colback=cellbackground, colframe=cellborder]
\prompt{In}{incolor}{31}{\boxspacing}
\begin{Verbatim}[commandchars=\\\{\}]
\PY{n}{pv\PYZus{}df} \PY{o}{=} \PY{n}{pd}\PY{o}{.}\PY{n}{DataFrame}\PY{p}{(}\PY{n}{pv}\PY{o}{.}\PY{n}{squeeze}\PY{p}{(}\PY{p}{)}\PY{o}{.}\PY{n}{to\PYZus{}series}\PY{p}{(}\PY{p}{)}\PY{p}{)}
\end{Verbatim}
\end{tcolorbox}

\begin{tcolorbox}[breakable, size=fbox, boxrule=1pt, pad at break*=1mm,colback=cellbackground, colframe=cellborder]
\prompt{In}{incolor}{32}{\boxspacing}
\begin{Verbatim}[commandchars=\\\{\}]
\PY{n}{tz} \PY{o}{=} \PY{n}{timezone}\PY{p}{(}\PY{l+s+s1}{\PYZsq{}}\PY{l+s+s1}{GMT}\PY{l+s+s1}{\PYZsq{}}\PY{p}{,} \PY{p}{)}
\PY{n}{pv\PYZus{}df}\PY{p}{[}\PY{l+s+s1}{\PYZsq{}}\PY{l+s+s1}{Date\PYZus{}Time}\PY{l+s+s1}{\PYZsq{}}\PY{p}{]} \PY{o}{=} \PY{n}{pv\PYZus{}df}\PY{o}{.}\PY{n}{index}\PY{o}{.}\PY{n}{values}
\PY{n}{pv\PYZus{}df}\PY{p}{[}\PY{l+s+s1}{\PYZsq{}}\PY{l+s+s1}{Date\PYZus{}Time}\PY{l+s+s1}{\PYZsq{}}\PY{p}{]} \PY{o}{=} \PY{n}{pv\PYZus{}df}\PY{p}{[}\PY{l+s+s1}{\PYZsq{}}\PY{l+s+s1}{Date\PYZus{}Time}\PY{l+s+s1}{\PYZsq{}}\PY{p}{]}\PY{o}{.}\PY{n}{apply}\PY{p}{(}\PY{n}{tz}\PY{o}{.}\PY{n}{localize}\PY{p}{,} \PY{n}{is\PYZus{}dst}\PY{o}{=}\PY{k+kc}{False}\PY{p}{)}
\PY{n}{pv\PYZus{}df}\PY{p}{[}\PY{l+s+s1}{\PYZsq{}}\PY{l+s+s1}{Date\PYZus{}Time}\PY{l+s+s1}{\PYZsq{}}\PY{p}{]} \PY{o}{=} \PY{n}{pv\PYZus{}df}\PY{p}{[}\PY{l+s+s1}{\PYZsq{}}\PY{l+s+s1}{Date\PYZus{}Time}\PY{l+s+s1}{\PYZsq{}}\PY{p}{]}\PY{o}{.}\PY{n}{dt}\PY{o}{.}\PY{n}{tz\PYZus{}convert}\PY{p}{(}\PY{n}{timezone}\PY{p}{(}\PY{l+s+s1}{\PYZsq{}}\PY{l+s+s1}{US/Central}\PY{l+s+s1}{\PYZsq{}}\PY{p}{)}\PY{p}{)}

\PY{n}{pv\PYZus{}df}\PY{o}{.}\PY{n}{set\PYZus{}index}\PY{p}{(}\PY{l+s+s1}{\PYZsq{}}\PY{l+s+s1}{Date\PYZus{}Time}\PY{l+s+s1}{\PYZsq{}}\PY{p}{,} \PY{n}{inplace}\PY{o}{=}\PY{k+kc}{True}\PY{p}{)}
\end{Verbatim}
\end{tcolorbox}

\hypertarget{plotting-in-depth-solar-vs.-ua-power-consumption-vs.-simple-solar-calculation}{%
\subsection{Plotting In-depth Solar vs.~UA Power Consumption vs.~Simple
Solar
Calculation}\label{plotting-in-depth-solar-vs.-ua-power-consumption-vs.-simple-solar-calculation}}

\hypertarget{average-weekly-power}{%
\subsubsection{Average Weekly Power}\label{average-weekly-power}}

\begin{center}
\adjustimage{max size={0.9\linewidth}{0.9\paperheight}}{Final Project_files/Final Project_63_1.png}
\end{center}
{ \hspace*{\fill} \\}

\hypertarget{daily-in-may}{%
\subsubsection{Daily in September}\label{daily-in-may}}

The plot below shows that the daily peaks in both solar power generation and UA's power consumption actually coincide in September.

\begin{center}
\adjustimage{max size={0.9\linewidth}{0.9\paperheight}}{Final Project_files/Final Project_65_0.png}
\end{center}
{ \hspace*{\fill} \\}
    
\hypertarget{determining-the-actual-area-required-for-solar-panels-to-power-uas-campus}{%
\subsubsection{Determining the actual area required for solar panels to
power UA's
campus}\label{determining-the-actual-area-required-for-solar-panels-to-power-uas-campus}}

\rmfamily
UA's average power consumption peaks in September at a value of 35675
kW. In order for solar to \textbf{completely} supply UA's campus with
power, the monthly average solar power would have to be greater than this
value. Assuming that both solar radiation and solar panel efficiency are
fixed, this calls for more area devoted to solar panels. The
calculations below determine roughly the area required, though the
simple calculation over-estimates the solar power available, and a
correction factor, $C_f$, will be applied, which is simply
\(\frac{\textrm{era5 estimate}_{\textrm{September}}}{\textrm{simple estimate}_{\textrm{September}}}\). The actual area required is calculated as

 \[A = \frac{P_{desired}}{N_{pv} \cdot G \cdot C_f}\]

The AC/DC convererter efficiency is not explicitly included in the formula as it is included in the $C_f$ calculation.

Its important to note that this calculation does not take into consideration the energy lost in electricity storage. If UA were to be completely powered by solar panels, there would have to be an energy storage system as well - which has its own inefficiencies.

\begin{tcolorbox}[breakable, size=fbox, boxrule=1pt, pad at break*=1mm,colback=cellbackground, colframe=cellborder]
\prompt{In}{incolor}{35}{\boxspacing}
\begin{Verbatim}[commandchars=\\\{\}]
\PY{n}{total\PYZus{}required\PYZus{}power} \PY{o}{=} \PY{n}{total\PYZus{}df}\PY{o}{.}\PY{n}{loc}\PY{p}{[}\PY{n}{total\PYZus{}df}\PY{o}{.}\PY{n}{index}\PY{o}{.}\PY{n}{month} \PY{o}{==} \PY{l+m+mi}{9}\PY{p}{,} \PY{l+s+s1}{\PYZsq{}}\PY{l+s+s1}{TOTAL \PYZhy{} KW\PYZhy{}TOT}\PY{l+s+s1}{\PYZsq{}}\PY{p}{]}\PY{o}{.}\PY{n}{pint}\PY{o}{.}\PY{n}{to}\PY{p}{(}\PY{l+s+s1}{\PYZsq{}}\PY{l+s+s1}{MW}\PY{l+s+s1}{\PYZsq{}}\PY{p}{)}\PY{o}{.}\PY{n}{mean}\PY{p}{(}\PY{p}{)}
\PY{n}{total\PYZus{}required\PYZus{}power}
\end{Verbatim}
\end{tcolorbox}
 
            
\prompt{Out}{outcolor}{35}{}
    
    $35.67570416666667\ \mathrm{megawatt}$

\hypertarget{correction-factor}{%
\paragraph{Correction Factor}\label{correction-factor}}

    \begin{tcolorbox}[breakable, size=fbox, boxrule=1pt, pad at break*=1mm,colback=cellbackground, colframe=cellborder]
\prompt{In}{incolor}{36}{\boxspacing}
\begin{Verbatim}[commandchars=\\\{\}]
\PY{n}{corr\PYZus{}factor} \PY{o}{=} \PY{n}{pv\PYZus{}df}\PY{p}{[}\PY{l+m+mi}{0}\PY{p}{]}\PY{o}{.}\PY{n}{loc}\PY{p}{[}\PY{n}{pv\PYZus{}df}\PY{o}{.}\PY{n}{index}\PY{o}{.}\PY{n}{month} \PY{o}{==} \PY{l+m+mi}{9}\PY{p}{]}\PY{o}{.}\PY{n}{mean}\PY{p}{(}\PY{p}{)} \PY{o}{/} \PY{n}{parking\PYZus{}lots}\PY{p}{[}\PY{l+s+sa}{f}\PY{l+s+s2}{\PYZdq{}}\PY{l+s+s2}{P\PYZus{}9}\PY{l+s+s2}{\PYZdq{}}\PY{p}{]}\PY{o}{.}\PY{n}{sum}\PY{p}{(}\PY{p}{)}\PY{o}{.}\PY{n}{to}\PY{p}{(}\PY{l+s+s1}{\PYZsq{}}\PY{l+s+s1}{MW}\PY{l+s+s1}{\PYZsq{}}\PY{p}{)}\PY{o}{.}\PY{n}{magnitude}
\PY{n}{corr\PYZus{}factor}
\end{Verbatim}
\end{tcolorbox}

            \begin{tcolorbox}[breakable, size=fbox, boxrule=.5pt, pad at break*=1mm, opacityfill=0]
\prompt{Out}{outcolor}{36}{\boxspacing}
\begin{Verbatim}[commandchars=\\\{\}]
0.677499185567263
\end{Verbatim}
\end{tcolorbox}
        
\hypertarget{the-total-area-required}{%
\subsubsection{The Total Area Required}\label{the-total-area-required}}

\begin{tcolorbox}[breakable, size=fbox, boxrule=1pt, pad at break*=1mm,colback=cellbackground, colframe=cellborder]
\prompt{In}{incolor}{37}{\boxspacing}
\begin{Verbatim}[commandchars=\\\{\}]
\PY{n}{A} \PY{o}{=} \PY{n}{total\PYZus{}required\PYZus{}power} \PY{o}{/} \PY{p}{(}\PY{n}{N\PYZus{}pv} \PY{o}{*} \PY{n}{G\PYZus{}solar}\PY{p}{[}\PY{l+m+mi}{9}\PY{p}{]}\PY{p}{[}\PY{l+s+s1}{\PYZsq{}}\PY{l+s+s1}{G}\PY{l+s+s1}{\PYZsq{}}\PY{p}{]} \PY{o}{*} \PY{n}{corr\PYZus{}factor}\PY{p}{)}
\PY{n}{A}\PY{o}{.}\PY{n}{to}\PY{p}{(}\PY{l+s+s1}{\PYZsq{}}\PY{l+s+s1}{m\PYZca{}2}\PY{l+s+s1}{\PYZsq{}}\PY{p}{)}
\end{Verbatim}
\end{tcolorbox}
 
            
\prompt{Out}{outcolor}{37}{}
    
$1396239.0932734653\ \mathrm{meter}^{2}$

\rmfamily

To put the amount of area required into perspective, the image below shows a box with equivalent area drawn on top of UA's main campus.

\begin{figure}
\centering
\includegraphics{Final Project_files/area.png}
\caption{area}
\end{figure}

\hypertarget{hydro-battery-calculations}{%
\subsection{Hydro Battery
Calculations}\label{hydro-battery-calculations}}

\rmfamily
I was going to do hydro-battery calculations for my proposal, as some
storage device is necessary to capture the full potential of solar power
if it is to be the main power source. Given UA's power demand though,
storage is not required. Even during peak-sun periods, the solar panels
do not provide excess power that needs to be stored.

If there was an excess energy provided by the panels, I think an
interesting storage solution would be a hydro-battery. UA's main campus
has two geological advantages for a hydro-battery: a river 500m from
campus and \textasciitilde100ft of elevation change between campus and
the river. The river provides ample water for a battery to operate with
relatively little impact down / upstream, with a flow rate of 2000 cubic
feet per second (from TODA1 gauge of \href{https://water.weather.gov/ahps2/hydrograph_to_xml.php?gage=toda1&output=tabular&time_zone=cst}{NWS's Advanced Hydrological Prediction Service}). The natural change in elevation means that water could
be pumped to a reservoir on UA's campus and stored with little investment
in infrastructure.

There is of course the glaring question as to why both the solar and
hydro-storage have to be on UA's campus and not instead on land else
where. The answer is simply that for the class we were asked to propose
a project related to the campus and using campus as the "land source" provided a good sense of scale for the area that solar power requires.


\hypertarget{economics-of-the-proposal}{%
\section{Economics of the Proposal}\label{economics-of-the-proposal}}

This section is dedicated to analysing the economics of implementing solar panels on UA's campus. As a reference point to the installation cost of solar, I looked up the cost of the new Tutwiler Hall being built on campus. Its current expected cost in \textbf{\$144,900,659.00} per Building Bama's website~\citep{building-bama}.

For the engineering economics calculations in this section, the assumed interest rate is 6\%. As a caveat, it is likely that the University of Alabama has access to lower interest rates, but I could not find better information on it.

\begin{tcolorbox}[breakable, size=fbox, boxrule=1pt, pad at break*=1mm,colback=cellbackground, colframe=cellborder]
\prompt{In}{incolor}{99}{\boxspacing}
\begin{Verbatim}[commandchars=\\\{\}]
\PY{n}{i} \PY{o}{=} \PY{l+m+mf}{0.06}
\end{Verbatim}
\end{tcolorbox}

\hypertarget{solar-implementation-cost}{%
\subsection{Solar Implementation
Cost}\label{solar-implementation-cost}}

\rmfamily
Per the Solar Energy Industries Association (SEIA), the average cost of utility-level solar projects in Q3 of 2021 is \$0.90 per watt of DC power~\cite{rumery_davis_2021}. This value is used for the cost analysis in this section, but it is likely that UA would have access to discounted rates (through government rebates) given that it is a public research institution.

\begin{tcolorbox}[breakable, size=fbox, boxrule=1pt, pad at break*=1mm,colback=cellbackground, colframe=cellborder]
\prompt{In}{incolor}{85}{\boxspacing}
\begin{Verbatim}[commandchars=\\\{\}]
\PY{n}{parking\PYZus{}lots}\PY{p}{[}\PY{l+s+s1}{\PYZsq{}}\PY{l+s+s1}{PV Capacity}\PY{l+s+s1}{\PYZsq{}}\PY{p}{]}\PY{o}{.}\PY{n}{sum}\PY{p}{(}\PY{p}{)}
\end{Verbatim}
\end{tcolorbox}

\begin{tcolorbox}[breakable, size=fbox, boxrule=.5pt, pad at break*=1mm, opacityfill=0]
\prompt{Out}{outcolor}{85}{\boxspacing}
\begin{Verbatim}[commandchars=\\\{\}]
23.404653
\end{Verbatim}
\end{tcolorbox}
        
\begin{tcolorbox}[breakable, size=fbox, boxrule=1pt, pad at break*=1mm,colback=cellbackground, colframe=cellborder]
\prompt{In}{incolor}{83}{\boxspacing}
\begin{Verbatim}[commandchars=\\\{\}]
\PY{n}{C\PYZus{}solar} \PY{o}{=} \PY{l+m+mf}{0.9} \PY{o}{/} \PY{n}{ureg}\PY{o}{.}\PY{n}{W}
\PY{n}{cost\PYZus{}solar} \PY{o}{=} \PY{n}{parking\PYZus{}lots}\PY{p}{[}\PY{l+s+s1}{\PYZsq{}}\PY{l+s+s1}{PV Capacity}\PY{l+s+s1}{\PYZsq{}}\PY{p}{]}\PY{o}{.}\PY{n}{sum}\PY{p}{(}\PY{p}{)} \PY{o}{*} \PY{n}{ureg}\PY{o}{.}\PY{n}{MW} \PY{o}{*} \PY{n}{C\PYZus{}solar}
\PY{n}{cost\PYZus{}solar} \PY{o}{=} \PY{n}{cost\PYZus{}solar}\PY{o}{.}\PY{n}{to}\PY{p}{(}\PY{l+s+s1}{\PYZsq{}}\PY{l+s+s1}{dimensionless}\PY{l+s+s1}{\PYZsq{}}\PY{p}{)}
\PY{n}{cost\PYZus{}solar}\PY{o}{.}\PY{n}{magnitude}
\end{Verbatim}
\end{tcolorbox}

\begin{tcolorbox}[breakable, size=fbox, boxrule=.5pt, pad at break*=1mm, opacityfill=0]
\prompt{Out}{outcolor}{83}{\boxspacing}
\begin{Verbatim}[commandchars=\\\{\}]
21064187.700000003
\end{Verbatim}
\end{tcolorbox}

\rmfamily        
The turn-key installation cost is \$21,064,187. In comparison to the cost of the new Tutwiler Hall construction, this would be a relatively small project for the university.

\hypertarget{cost-of-electricity}{%
\subsection{Cost of Electricity}\label{cost-of-electricity}}

Using my monthly power bill in Tuscaloosa, Alabama, I found the cost of electricity to be \$0.1353 per kWh. It is likely that UA receives 
discounted rates from the local utility company.
Without having the information on UAs effective cost of electricity, I use the consumer cost of electricity.

\begin{tcolorbox}[breakable, size=fbox, boxrule=1pt, pad at break*=1mm,colback=cellbackground, colframe=cellborder]
\prompt{In}{incolor}{86}{\boxspacing}
\begin{Verbatim}[commandchars=\\\{\}]
\PY{n}{c\PYZus{}p\PYZus{}kwh} \PY{o}{=} \PY{l+m+mf}{13.53} \PY{o}{/} \PY{l+m+mi}{100} \PY{o}{/} \PY{n}{ureg}\PY{o}{.}\PY{n}{kWh}
\end{Verbatim}
\end{tcolorbox}

\hypertarget{determining-uas-yearly-power-bill}{%
\subsection{Determining UA's Yearly Power
Bill}\label{determining-uas-yearly-power-bill}}

\begin{tcolorbox}[breakable, size=fbox, boxrule=1pt, pad at break*=1mm,colback=cellbackground, colframe=cellborder]
\prompt{In}{incolor}{87}{\boxspacing}
\begin{Verbatim}[commandchars=\\\{\}]
\PY{n}{yearly\PYZus{}elec} \PY{o}{=} \PY{n}{total\PYZus{}df}\PY{p}{[}\PY{l+s+s1}{\PYZsq{}}\PY{l+s+s1}{TOTAL  Energy}\PY{l+s+s1}{\PYZsq{}}\PY{p}{]}\PY{o}{.}\PY{n}{sum}\PY{p}{(}\PY{p}{)}
\end{Verbatim}
\end{tcolorbox}

\begin{tcolorbox}[breakable, size=fbox, boxrule=1pt, pad at break*=1mm,colback=cellbackground, colframe=cellborder]
\prompt{In}{incolor}{88}{\boxspacing}
\begin{Verbatim}[commandchars=\\\{\}]
\PY{n}{yearly\PYZus{}elec}\PY{o}{.}\PY{n}{to}\PY{p}{(}\PY{l+s+s1}{\PYZsq{}}\PY{l+s+s1}{kWh}\PY{l+s+s1}{\PYZsq{}}\PY{p}{)} \PY{o}{*} \PY{n}{c\PYZus{}p\PYZus{}kwh}
\end{Verbatim}
\end{tcolorbox}
 
            
\prompt{Out}{outcolor}{88}{}
    
$33034051.522725$

\rmfamily
Using the assumptions alluded to above, UA pays \$33,034,051 per year in
electricity costs.

\hypertarget{the-total-energy-that-solar-could-provide-ua}{%
\subsection{Solar's Yearly Savings}\label{the-total-energy-that-solar-could-provide-ua}}

\rmfamily
To find the total energy that the solar panels would provide in a year, the numerical integral of the atlite power estimate throughout the entire year was calculated.

\begin{tcolorbox}[breakable, size=fbox, boxrule=1pt, pad at break*=1mm,colback=cellbackground, colframe=cellborder]
\prompt{In}{incolor}{89}{\boxspacing}
\begin{Verbatim}[commandchars=\\\{\}]
\PY{n}{yearly\PYZus{}solar\PYZus{}electricity} \PY{o}{=} \PY{p}{(}\PY{n}{np}\PY{o}{.}\PY{n}{trapz}\PY{p}{(}\PY{n}{pv\PYZus{}df}\PY{p}{[}\PY{l+m+mi}{0}\PY{p}{]}\PY{p}{,} \PY{n}{dx}\PY{o}{=}\PY{l+m+mi}{3600}\PY{p}{)} \PY{o}{*} \PY{n}{ureg}\PY{o}{.}\PY{n}{MW} \PY{o}{*} \PY{n}{ureg}\PY{o}{.}\PY{n}{s}\PY{p}{)}\PY{o}{.}\PY{n}{to}\PY{p}{(}\PY{l+s+s1}{\PYZsq{}}\PY{l+s+s1}{kWh}\PY{l+s+s1}{\PYZsq{}}\PY{p}{)}
\end{Verbatim}
\end{tcolorbox}

\rmfamily
Multiplying the amount of solar energy that the parking lots would capture by the cost of electricity, the yearly savings to UA can be calculated.

\begin{tcolorbox}[breakable, size=fbox, boxrule=1pt, pad at break*=1mm,colback=cellbackground, colframe=cellborder]
\prompt{In}{incolor}{90}{\boxspacing}
\begin{Verbatim}[commandchars=\\\{\}]
\PY{n}{yearly\PYZus{}solar\PYZus{}electricity\PYZus{}savings} \PY{o}{=} \PY{n}{yearly\PYZus{}solar\PYZus{}electricity}\PY{o}{.}\PY{n}{to}\PY{p}{(}\PY{l+s+s1}{\PYZsq{}}\PY{l+s+s1}{kWh}\PY{l+s+s1}{\PYZsq{}}\PY{p}{)} \PY{o}{*} \PY{n}{c\PYZus{}p\PYZus{}kwh}
\end{Verbatim}
\end{tcolorbox}

\rmfamily
Converting the parking lots to solar installations could save UA \$4,407,000.00 per year on electricity.

\hypertarget{present-value-of-annual-savings}{%
\subsection{Present Value of Annual
Savings}\label{present-value-of-annual-savings}}

\rmfamily
The present value calculation relies on the installation lifetime. I could not find great data on SunPower's website for the actual lifetime of the panels, but they have a 25 year warranty. Going forward, I will use 25 years as the lifetime of the project, $n$.

The formula for present value of annual savings, $P$, is below

\[P = U \left[\frac{1 - (1 + i)^{-n}}{i}\right]\]

where $U$ is the yearly savings calculated in Section~\ref{the-total-energy-that-solar-could-provide-ua}, and $i$ is the interest rate. 

\begin{tcolorbox}[breakable, size=fbox, boxrule=1pt, pad at break*=1mm,colback=cellbackground, colframe=cellborder]
\prompt{In}{incolor}{101}{\boxspacing}
\begin{Verbatim}[commandchars=\\\{\}]
\PY{n}{n} \PY{o}{=} \PY{l+m+mi}{25}

\PY{n}{pv\PYZus{}as} \PY{o}{=} \PY{n}{yearly\PYZus{}solar\PYZus{}electricity\PYZus{}savings}\PY{o}{.}\PY{n}{magnitude} \PY{o}{*} \PY{p}{(}\PY{l+m+mi}{1} \PY{o}{\PYZhy{}} \PY{p}{(}\PY{l+m+mi}{1} \PY{o}{+} \PY{n}{i}\PY{p}{)} \PY{o}{*}\PY{o}{*}\PY{p}{(}\PY{o}{\PYZhy{}}\PY{l+m+mi}{1} \PY{o}{*} \PY{n}{n}\PY{p}{)}\PY{p}{)} \PY{o}{/} \PY{n}{i}
\PY{n}{pv\PYZus{}as}
\end{Verbatim}
\end{tcolorbox}

\begin{tcolorbox}[breakable, size=fbox, boxrule=.5pt, pad at break*=1mm, opacityfill=0]
\prompt{Out}{outcolor}{101}{\boxspacing}
\begin{Verbatim}[commandchars=\\\{\}]
56339815.92712577
\end{Verbatim}
\end{tcolorbox}
        
\hypertarget{present-value-of-the-maintence-costs}{%
\subsubsection{Present Value of the Maintenance
Costs}\label{present-value-of-the-maintence-costs}}

Just as there is a present value of the savings, there is also a present value of the maintenance costs of the solar project. Solar maintenance mainly involves cleaning the panels and I found an estimate online for \$11.50 per kWh~\citep{homeguide}.  

\begin{tcolorbox}[breakable, size=fbox, boxrule=1pt, pad at break*=1mm,colback=cellbackground, colframe=cellborder]
\prompt{In}{incolor}{50}{\boxspacing}
\begin{Verbatim}[commandchars=\\\{\}]
\PY{n}{C\PYZus{}repair} \PY{o}{=} \PY{n}{parking\PYZus{}lots}\PY{p}{[}\PY{l+s+s1}{\PYZsq{}}\PY{l+s+s1}{PV Capacity}\PY{l+s+s1}{\PYZsq{}}\PY{p}{]}\PY{o}{.}\PY{n}{sum}\PY{p}{(}\PY{p}{)} \PY{o}{*} \PY{n}{ureg}\PY{o}{.}\PY{n}{MW} \PY{o}{*} \PY{l+m+mf}{11.50} \PY{o}{/} \PY{n}{ureg}\PY{o}{.}\PY{n}{kW}
\PY{n}{C\PYZus{}repair} \PY{o}{=} \PY{n}{C\PYZus{}repair}\PY{o}{.}\PY{n}{to}\PY{p}{(}\PY{l+s+s1}{\PYZsq{}}\PY{l+s+s1}{dimensionless}\PY{l+s+s1}{\PYZsq{}}\PY{p}{)}
\end{Verbatim}
\end{tcolorbox}

\begin{tcolorbox}[breakable, size=fbox, boxrule=1pt, pad at break*=1mm,colback=cellbackground, colframe=cellborder]
\prompt{In}{incolor}{102}{\boxspacing}
\begin{Verbatim}[commandchars=\\\{\}]
\PY{n}{pv\PYZus{}repair} \PY{o}{=} \PY{n}{C\PYZus{}repair}\PY{o}{.}\PY{n}{magnitude} \PY{o}{*} \PY{p}{(}\PY{l+m+mi}{1} \PY{o}{\PYZhy{}} \PY{p}{(}\PY{l+m+mi}{1} \PY{o}{+} \PY{n}{i}\PY{p}{)} \PY{o}{*}\PY{o}{*}\PY{p}{(}\PY{o}{\PYZhy{}}\PY{l+m+mi}{1} \PY{o}{*} \PY{n}{n}\PY{p}{)}\PY{p}{)} \PY{o}{/} \PY{n}{i}
\PY{n}{pv\PYZus{}repair}
\end{Verbatim}
\end{tcolorbox}

\begin{tcolorbox}[breakable, size=fbox, boxrule=.5pt, pad at break*=1mm, opacityfill=0]
\prompt{Out}{outcolor}{102}{\boxspacing}
\begin{Verbatim}[commandchars=\\\{\}]
3440685.1731863804
\end{Verbatim}
\end{tcolorbox}
        
\hypertarget{net-present-value}{%
\subsubsection{Net Present Value}\label{net-present-value}}

\begin{tcolorbox}[breakable, size=fbox, boxrule=1pt, pad at break*=1mm,colback=cellbackground, colframe=cellborder]
\prompt{In}{incolor}{105}{\boxspacing}
\begin{Verbatim}[commandchars=\\\{\}]
\PY{n}{npv} \PY{o}{=} \PY{n}{pv\PYZus{}as} \PY{o}{\PYZhy{}} \PY{n}{pv\PYZus{}repair}
\PY{n}{npv}
\end{Verbatim}
\end{tcolorbox}

\begin{tcolorbox}[breakable, size=fbox, boxrule=.5pt, pad at break*=1mm, opacityfill=0]
\prompt{Out}{outcolor}{105}{\boxspacing}
\begin{Verbatim}[commandchars=\\\{\}]
52899130.75393939
\end{Verbatim}
\end{tcolorbox}
        
\rmfamily
The net present value of the project is the difference between the
present value of savings the present value of repair. For this project,
the net present value is \$52.9 million.

\hypertarget{roi}{%
\subsection{ROI}\label{roi}}

\begin{tcolorbox}[breakable, size=fbox, boxrule=1pt, pad at break*=1mm,colback=cellbackground, colframe=cellborder]
\prompt{In}{incolor}{112}{\boxspacing}
\begin{Verbatim}[commandchars=\\\{\}]
\PY{n}{roi} \PY{o}{=} \PY{p}{(}\PY{n}{yearly\PYZus{}solar\PYZus{}electricity\PYZus{}savings} \PY{o}{\PYZhy{}} \PY{n}{C\PYZus{}repair}\PY{p}{)} \PY{o}{/} \PY{n}{cost\PYZus{}solar}
\PY{n}{roi}\PY{o}{.}\PY{n}{magnitude} \PY{o}{*} \PY{l+m+mi}{100}
\end{Verbatim}
\end{tcolorbox}

            \begin{tcolorbox}[breakable, size=fbox, boxrule=.5pt, pad at break*=1mm, opacityfill=0]
\prompt{Out}{outcolor}{112}{\boxspacing}
\begin{Verbatim}[commandchars=\\\{\}]
19.645312005848915
\end{Verbatim}
\end{tcolorbox}
        
The return on investment is 19.6\%, which would be generally thought of as a good investment.

\hypertarget{payback-period}{%
\subsection{Payback Period}\label{payback-period}}

\rmfamily
Knowing the up-front cost of the solar panels, the yearly electric
savings, and the yearly solar panel maintenance cost, we can calculate
the payback period, $n_{dpb}$, with the equation below.

\[n_{dpb} = \frac{\log \left[1 - (\frac{P}{U})i\right]^{-1}}{\log(1 + i)}\]

\begin{tcolorbox}[breakable, size=fbox, boxrule=1pt, pad at break*=1mm,colback=cellbackground, colframe=cellborder]
\prompt{In}{incolor}{79}{\boxspacing}
\begin{Verbatim}[commandchars=\\\{\}]
\PY{n}{P} \PY{o}{=} \PY{n}{cost\PYZus{}solar}\PY{o}{.}\PY{n}{magnitude}
\PY{n}{U} \PY{o}{=} \PY{p}{(}\PY{n}{yearly\PYZus{}solar\PYZus{}electricity\PYZus{}savings} \PY{o}{\PYZhy{}} \PY{n}{C\PYZus{}repair}\PY{p}{)}\PY{o}{.}\PY{n}{magnitude}
\PY{n}{i} \PY{o}{=} \PY{l+m+mf}{0.06}
\end{Verbatim}
\end{tcolorbox}

\begin{tcolorbox}[breakable, size=fbox, boxrule=1pt, pad at break*=1mm,colback=cellbackground, colframe=cellborder]
\prompt{In}{incolor}{80}{\boxspacing}
\begin{Verbatim}[commandchars=\\\{\}]
\PY{n}{n\PYZus{}dpb} \PY{o}{=} \PY{n}{np}\PY{o}{.}\PY{n}{log}\PY{p}{(}\PY{p}{(}\PY{l+m+mi}{1} \PY{o}{\PYZhy{}} \PY{p}{(}\PY{n}{P} \PY{o}{/} \PY{n}{U}\PY{p}{)} \PY{o}{*} \PY{n}{i}\PY{p}{)} \PY{o}{*}\PY{o}{*} \PY{p}{(}\PY{o}{\PYZhy{}}\PY{l+m+mi}{1}\PY{p}{)}\PY{p}{)} \PY{o}{/} \PY{n}{np}\PY{o}{.}\PY{n}{log}\PY{p}{(}\PY{l+m+mi}{1} \PY{o}{+} \PY{n}{i}\PY{p}{)}
\end{Verbatim}
\end{tcolorbox}

\begin{tcolorbox}[breakable, size=fbox, boxrule=1pt, pad at break*=1mm,colback=cellbackground, colframe=cellborder]
\prompt{In}{incolor}{81}{\boxspacing}
\begin{Verbatim}[commandchars=\\\{\}]
\PY{n}{n\PYZus{}dpb}
\end{Verbatim}
\end{tcolorbox}

\begin{tcolorbox}[breakable, size=fbox, boxrule=.5pt, pad at break*=1mm, opacityfill=0]
\prompt{Out}{outcolor}{81}{\boxspacing}
\begin{Verbatim}[commandchars=\\\{\}]
6.254497071824046
\end{Verbatim}
\end{tcolorbox}
        
The projects payback period is 6.25 years using the discounted payback period equation.

\hypertarget{final-results}{%
\subsection{Final Results}\label{final-results}}

\begin{tcolorbox}[breakable, size=fbox, boxrule=1pt, pad at break*=1mm,colback=cellbackground, colframe=cellborder]
\prompt{In}{incolor}{148}{\boxspacing}
\begin{Verbatim}[commandchars=\\\{\}]
\PY{c+c1}{\PYZsh{}\PYZsh{} Alabama\PYZsq{}s Carbon Intensity: From https://www.eia.gov/environment/emissions/state/}
\PY{n}{al\PYZus{}co2} \PY{o}{=} \PY{p}{(}\PY{l+m+mf}{47.6423949721341} \PY{o}{*} \PY{n}{ureg}\PY{o}{.}\PY{n}{kg} \PY{o}{/} \PY{p}{(}\PY{l+m+mf}{1e6} \PY{o}{*} \PY{n}{ureg}\PY{o}{.}\PY{n}{BTU}\PY{p}{)}\PY{p}{)}\PY{o}{.}\PY{n}{to}\PY{p}{(}\PY{l+s+s1}{\PYZsq{}}\PY{l+s+s1}{kg/kWh}\PY{l+s+s1}{\PYZsq{}}\PY{p}{)}
\PY{n}{al\PYZus{}co2}
\end{Verbatim}
\end{tcolorbox}
 
            
\prompt{Out}{outcolor}{148}{}
$0.16256\ \frac{\mathrm{kilogram}}{\mathrm{kilowatt\_hour}}$

\rmfamily
The US Energy Information Administration publishes the carbon intensity of every states electricity grid~\citep{state-carbon}. By using Alabama's, I can estimate the carbon emissions that UA would save by adding solar panels. The saved carbon emissions are displayed in the table below, which also appears in the Introduction~\ref{introduction}.

The following table has been generated to summarize the project. It's
important to note that this table is generated via the Jupyter code
cells not shown, and won't execute where it is at. Probably not great
programming practice, but it had to be in the introduction to satisfy
the project requirements.

{\footnotesize
\begin{tabular}{llllllll}
% \small
% \addtolength{\tabcolsep}{-1pt}
\toprule
{} &    Cost [\$] & Electric Cost [\$] & CO2 [Mton] & Energy [GWh] & Power [MW] & Payback Per. &    ROI [\%] \\
\midrule
No Solar &           - &               33,034,052 &        39,690,322 &                          244 &                         28 &              - &          - \\
Solar    &  21,064,188 &               25,186,087 &        34,394,988 &                          212 &                         24 &           6.25 &  19.65 \\
\bottomrule
\end{tabular}
}

        
\hypertarget{conclusion}{%
\section{Conclusion}\label{conclusion}}

The amount of electricity that UA's campus uses is striking. Comparing it to solar area really puts it into perspective. I'm sure that UA has considered energy saving efforts before, but as I write this I am sitting in a library that is cooled into the low 60s. That feels excessive, but it is also beyond the scope of this project.

Replacing UA's on-campus parking with solar is a project that was doomed to fail from the start. Its not so much cost prohibitive as it is public-approval prohibitive. That being said, adding utility-level solar to UA's campus would be a good investment. Alabama gets sufficient solar radiation to make large investment in a solar project worthwhile. Depending on the levels of government grants, incentives, etc., the cost could even be quite a bit less than my estimates. If UA had access to equivalent plots of unused land it's almost a no-brainer. The publicity from a solar project of this magnitude would likely drive funding to faculty as well, essentially reducing the cost of the project further.

While solar would be a good investment, the analysis in this project proves that completely decarbonizing the power sector is a daunting task. Solar is one of the most mature renewable energy technologies, and to power UAs campus with it would require a footprint the size of UAs campus itself (see Section~\ref{determining-the-actual-area-required-for-solar-panels-to-power-uas-campus}). It is going to take clever solutions to free even the University of Alabama from coal / natural-gas fired power plants. I am confident that humanity will solve the problem, but we need to see more investment and experimentation from the government and private sector.

\newpage
\bibliographystyle{plainnat}
\bibliography{bibfile}
    
\end{document}
